%%%%%%%%%%%%%%%%%%%%%%%%%%%%%%%%%%%%%%%%%
% "ModernCV" CV and Cover Letter
% LaTeX Template
% Version 1.3 (29/10/16)
%
% This template has been downloaded from:
% http://www.LaTeXTemplates.com
%
% Xavier Danaux (xdanaux@gmail.com) with modifications by:
% Original author:
% Vel (vel@latextemplates.com)
%
% License:
% CC BY-NC-SA 3.0 (http://creativecommons.org/licenses/by-nc-sa/3.0/)
%
% Important note:
% This template requires the moderncv.cls and .sty files to be in the same
% directory as this .tex file. These files provide the resume style and themes
% used for structuring the document.
%
%%%%%%%%%%%%%%%%%%%%%%%%%%%%%%%%%%%%%%%%%

%----------------------------------------------------------------------------------------
%	PACKAGES AND OTHER DOCUMENT CONFIGURATIONS
%----------------------------------------------------------------------------------------

\documentclass[10pt,a4paper,roman]{moderncv} % Font sizes: 10, 11, or 12; paper sizes: a4paper, letterpaper, a5paper, legalpaper, executivepaper or landscape; font families: sans or roman
% \nopagenumbers{} % suppress page numbers

\moderncvstyle{classic} % CV theme - options include: 'casual' (default), 'classic', 'oldstyle' and 'banking'
\moderncvcolor{blue} % CV color - options include: 'blue' (default), 'orange', 'green', 'red', 'purple', 'grey' and 'black'
\usepackage{multicol}
\usepackage{lipsum} % Used for inserting dummy 'Lorem ipsum' text into the template
\usepackage[hyperref]{}
\usepackage[scale=.88]{geometry} % Reduce document margins % scale=0.88
\setlength{\hintscolumnwidth}{2.2cm} % Uncomment to change the width of the dates column
% \setlength{\makecvtitlenamewidth}{10cm} % For the 'classic' style, uncomment to adjust the width of the space allocated to your name

% \usepackage[
%     backend=biber,
%     style=apa,
%     natbib=true,
%     sorting=ydnt
% ]{biblatex}
% \addbibresource{bibliography.bib}

% %% make my name bold
% {
% \renewcommand*{\mkbibnamegiven}[1]{%
%   \ifitemannotation{me}
%     {\textbf{#1}}
%     {#1}}
% \renewcommand*{\mkbibnamefamily}[1]{%
%   \ifitemannotation{me}
%     {\textbf{#1}}
%     {#1}
% }

\newcommand{\OA}{\textcolor{color2}{[OA]} }

%----------------------------------------------------------------------------------------
%	NAME AND CONTACT INFORMATION SECTION
%----------------------------------------------------------------------------------------

\firstname{Fabian C.} % Your first name
\familyname{Moss} % Your last name

% All information in this block is optional, comment out any lines you don't need
\title{Curriculum Vitae}
\address{Chemin de Renens 18}{CH-1004 Lausanne}
\mobile{(+41) 78 700 8485}
% \phone{(000) 111 1112}
%\fax{(000) 111 1113}
\email{fabian.moss@epfl.ch}
% \email{asinha@mt.iitr.ac.in}
\homepage{fabian-moss.de}{fabian-moss.de} % The first argument is the url for the clickable link, the second argument is the url displayed in the template - this allows special characters to be displayed such as the tilde in this example
\extrainfo{\href{https://twitter.com/fabianmoss}{{@}fabianmoss}}
% \photo[70pt][0.4pt]{pictures/picture} % The first bracket is the picture height, the second is the thickness of the frame around the picture (0pt for no frame)
% \quote{"A witty and playful quotation" - John Smith}

%----------------------------------------------------------------------------------------

\begin{document}

%----------------------------------------------------------------------------------------
%	COVER LETTER
%----------------------------------------------------------------------------------------

% To remove the cover letter, comment out this entire block

% \recipient{HR Department}{Corporation\\123 Pleasant Lane\\12345 City, State} % Letter recipient
% \date{\today} % Letter date
% \opening{Dear Sir or Madam,} % Opening greeting
% \closing{Sincerely yours,} % Closing phrase
% \enclosure[Attached]{curriculum vit\ae{}} % List of enclosed documents

% \makelettertitle % Print letter title

% \lipsum[1-2] % Dummy text
% \lipsum[4] % Dummy text

% \makeletterclosing % Print letter signature

% \newpage

%----------------------------------------------------------------------------------------
%	CURRICULUM VITAE
%----------------------------------------------------------------------------------------

\makecvtitle % Print the CV title

%----------------------------------------------------------------------------------------
%	EMPLOYMENT SECTION
%----------------------------------------------------------------------------------------

\section{Employment}

\cventry{2020--today}{Postdoctoral Researcher}{\newline {\'E}cole Polytechnique F{\'e}d{\'e}rale de Lausanne (EPFL)}{Digital and Cognitive Musicology Lab (DCML)}{}{Lausanne, Switzerland}  % Arguments not required can be left empty
\cventry{2017--2019}{Doctoral Assistant}{\newline{\'E}cole Polytechnique F{\'e}d{\'e}rale de Lausanne (EPFL)}{Digital and Cognitive Musicology Lab (DCML)}{}{Lausanne, Switzerland}  % Arguments not required can be left empty
\cventry{2015--2017}{Doctoral Assistant}{\newline Technische Universit{\"a}t Dresden (TUD)}{Dresden Music Cognition Lab (DMCL)}{}{Dresden, Germany}  % Arguments not required can be left empty
\cventry{2012--2014}{Conductor and vocal coach for children's choirs}{\newline Musikschule Leverkusen}{Project ``Jedem Kind seine Stimme'' (JEKISS)}{}{Leverkusen, Germany}  % Arguments not required can be left empty

%----------------------------------------------------------------------------------------
%	EDUCATION SECTION
%----------------------------------------------------------------------------------------

\section{Education}

\cventry{2017--2019}{{\'E}cole Polytechnique F{\'e}d{\'e}rale de Lausanne (EPFL)}{\newline Digital and Cognitive Musicology Lab (DCML)}{PhD student}{}{Lausanne, Switzerland}  % Arguments not required can be left empty
\cventry{Jan--Mar 2016}{Massachusetts Institute of Technology (MIT)}{\newline Department of Linguistics and Philosophy}{Visiting Student}{}{Cambridge, MA, USA}%}
\cventry{2015--2017}{Technische Universit{\"a}t Dresden (TUD)}{\newline Dresden Music Cognition Lab (DMCL)}{PhD student}{}{Dresden, Germany}  % Arguments not required can be left empty
\cventry{Jan--Apr 2012}{Escola Superior de Musica de Catalunya (ESMUC)}{}{\newline ERASMUS Exchange Student}{}{Barcelona, Spain}
\cventry{2011--2013}{Hochschule f{\"u}r Musik und Tanz K{\"o}ln (HfMT)}{\newline Musicology}{Master of Arts}{}{Cologne, Germany}  % Arguments not required can be left empty
\cventry{2008--2013}{Hochschule f{\"u}r Musik und Tanz K{\"o}ln (HfMT)}{\newline Music Education (Piano Major)}{Staatexamen [State Examination]}{}{Cologne, Germany}  % Arguments not required can be left empty
\cventry{2006--2016}{Universit{\"a}t zu K{\"o}ln (UzK)}{\newline Mathematics and Educational Sciences}{Staatexamen [State Examination]}{}{Cologne, Germany}  % Arguments not required can be left empty
\cventry{2002--2005}{Friedrich-Wilhelm-Gymnasium K{\"o}ln (FWG)}{}{\newline Abitur [German High School Diploma]}{}{Cologne, Germany}  % Arguments not required can be left empty
\cventry{1996--2002}{Europaschule K{\"o}ln, Gesamtschule Zollstock}{}{}{}{Cologne, Germany}
\cventry{1992--1996}{Katholische Grundschule Trierer Stra\ss e}{}{}{}{Cologne, Germany}

% \newpage
%----------------------------------------------------------------------------------------
%	TALKS SECTION
%----------------------------------------------------------------------------------------

\section{Talks, Conference Presentations, Posters, and Workshops}

\subsection{Workshops}

%% 2020
\cvitem{2020}{\textbf{Moss, F. C.} \emph{Data-Driven Music History}. International Conference of Students of Systematic Musicology, York University, September 14, 2020, York, UK [online].}

%% 2018
\cvitem{2018}{\textbf{Moss, F. C.} \emph{Corpus Research in Digital Musicology}. Seminar ``Willkommen in der Matrix: Digitale Anwendungen f{\"u}r die Musikanalyse in Theorie und Praxis'', University of Basel, Basel, Switzerland.}

\subsection{Talks}

%2021
% \cvitem{2021}{\textbf{Moss, F. C.} \& Markus Neuwirth. \emph{Die Entwicklung der tonalen Sprache in Beethovens Streichquartetten: Eine vergleichende Korpusstudie der Schaffensphasen}. XVII. Internationaler Kongress der Gesellschaft für Musikforschung, Universität Bonn, Abteilung für Musikwissenschaft/Sound Studies und Beethoven-Archiv des Beethoven-Hauses Bonn
% Bonn, Germany, September 28 -- October 1 2021, Bonn, Germany.}
% \cvitem{}{\textbf{Moss, F. C.} \emph{Discovering the line of fifths in a large historical corpus}. Future Directions of Music Cognition, The Ohio State University, May 10--14, 2020, Columbus, OH.}

% 2020
% \cvitem{2020}{\textbf{Moss, F. C.} \emph{TITLE} Round-table on ``Probability and Music'', 5th International Congress of Music and Mathematics (MusMat 2020) -- Perspectives and Applications of Mathematics in Post-Tonal Theories («Homage to Jamary Oliveira»),
% December 8--12, Rio de Janeiro, Brazil [online].}
\cvitem{2020}{\textbf{Moss, F. C.} \emph{Analyzing musical pieces on the Tonnetz using the \emph{pitchplots} Python library}. 20. Jahreskongress der Gesellschaft für Musiktheorie
(GMTH), Hochschule für Musik Detmold, October 1--4, 2020, Detmold, Germany [online].}
\cvitem{}{\textbf{Moss, F. C.} \emph{Computational Musicology and the Digital Humanities: Problems, Practices, and Prospects}. CRETA-Werkstatt \#9, Center for Reflected Text Analytics, University of Stuttgart, February 18, 2020, Stuttgart, Germany.}

% 2019
\cvitem{2019}{\textbf{Moss, F. C.} \emph{Transitions of Tonality: Perspectives on the Historical Changes of Tonal Pitch Relations from Computational Musicology, Music Theory, and the Digital Humanities}. University of Cologne, November 29, 2019, Cologne, Germany.}
\cvitem{}{\textbf{Moss, F. C.} \emph{Tracing the History of Tonality with Note Distributions}. ``Corpus Research as a Means of Unlocking Musical Grammar'' International Research Workshop, July 1--4, 2019, Tel-Aviv, Israel.}
\cvitem{}{\textbf{Moss, F. C.} \emph{Inferring Tonality from Note Distributions -- Why Models Matter (Poster)}. SEMPRE Graduate Conference 2019, Cambridge, UK.}
\cvitem{}{\textbf{Moss, F. C.} \emph{Analyzing Tonality with Note Distributions}. First Swiss Digital Humanities Student Exchange DHX2019, Basel, Switzerland. }

% 2018
\cvitem{2018}{\textbf{Moss, F. C.}, Souza, W. F. \& Rohrmeier, M. \emph{Harmony and Form in Brazilian Choro: A Corpus Study}. 15th International Conference on Music Perception and Cognition \& 10th triennial conference of the European Society for the Cognitive Sciences of Music, Graz, Austria. }
\cvitem{}{Aitken, C., O'Donnell, T. \& Rohrmeier, M. [Poster presented by \textbf{Moss, F. C.}]. \emph{A Maximum Likelihood Model for the Harmonic Analysis of Symbolic Music}. 15th Sound and Music Computing Conference ``Sonic Crossings''. Limassol, Cyprus.}
\cvitem{}{\textbf{Moss, F. C.} \emph{Corpus Research in Digital Musicology}. Seminar ``Willkommen in der Matrix: Digitale Anwendungen f{\"u}r die Musikanalyse in Theorie und Praxis'', University of Basel, Basel, Switzerland.}
\cvitem{}{Harasim, D., \textbf{Moss, F. C.} \& Ramirez, M. \emph{A Brief History of Tonality} (Poster). Applied Machine Learning Days, EPFL, Switzerland.}

% 2017
\cvitem{2017}{\textbf{Moss, F. C.} \emph{Formal Grammars and Ambiguity in Extended Tonality}. Workshop and Symposium on Schenkerian Analysis ``Wege der Kreativit{\"a}t -- Zwischen Erfindung und Rekonstruktion'', Universit{\"a}t der K{\"u}nste, Berlin, Germany.}
\cvitem{}{\textbf{Moss, F. C.}, Souza, W. F. \& Rohrmeier, M. \emph{Brazilian Choro: A New Data Set of Chord Transcriptions and Analyses of Harmonic and Formal Features}. 17. Jahreskongress der Gesellschaft f{\"u}r Musiktheorie (GMTH) \& 27. Arbeitstagung der Gesellschaft f{\"u}r Popularmusikforschung (GfPM) ``Popul{\"a}re Musik und ihre Theorien: Begegnungen -- Perspektivwechsel -- Transfers'', Graz, Austria. }
\cvitem{}{\textbf{Moss, F. C.}, Harasim, D., Neuwirth, M. \& Rohrmeier, M. \emph{Beethovens Streichquartette -- ein XML-basierter Korpus harmonischer Analysen in einem neuen Annotationssystem}. Jahrestagung der Gesellschaft f{\"u}r Musikforschung, Kassel, Germany.}
\cvitem{}{\textbf{Moss, F. C.}, Rohrmeier, M. \emph{Integrating Transformational and Hierarchical Models of Extended Tonality}. 9th European Music Analysis Conference (EuroMAC), Strasbourg, France.}
\cvitem{}{Rom, U., Je\ss ulat, A., \textbf{Moss, F. C.} \& Guter, I. \emph{Ambiguity, Illusion \& Timelessness in Late and Post-Tonal Harmony}. Panel discussion at the 9th European Music Analysis Conference (EuroMAC), Strasbourg, France.}
\cvitem{}{\textbf{Moss, F. C.} \emph{Musik und Sprache}. Talk for Student Association ``Denkzettel'', TUD, Dresden, Germany.}
\cvitem{}{\textbf{Moss, F. C.}, Rohrmeier, M. \& Bravo, F. \emph{Emotional Associations Evoked by Structural Properties of Musical Scales and Abstract Visual Shapes}. KOSMOS Dialogue ``Music, Emotion, and Visual Imagery'', Berlin, Germany.}
\cvitem{}{Harasim, D., \textbf{Moss, F. C.}, Neuwirth, M. \& Rohrmeier M. \emph{Beethoven's String Quartets: Introducing an XML-Based Corpus of Harmonic Labels Using a New Annotation System}. Music Encoding Conference, Tours, France.}

% 2016
\cvitem{2016}{\textbf{Moss, F. C.} \emph{Extended Tonality: Theoretical Challenges and their Relation to the Neuroscientific Study of Musical Syntax}. Max Planck Institute for Human Cognitive and Brain Sciences, Leipzig, Germany. }
\cvitem{}{\textbf{Moss, F. C.}, Rohrmeier, M. \emph{Structural Ambiguities in Language and Music} (Poster). Helsinki Summer School for Cognitive Neuroscience 2016 (HSSCN 2016).}
\cvitem{}{\textbf{Moss, F. C.}, Rohrmeier, M.  \emph{A grammatical approach to tension-resolution patterns in extended tonal harmony}. Meeting of the Computational Cognitive Science Group, Massachusetts Institute of Technology, Department of Brain and Cognitive Sciences, Cambridge, USA.}
\cvitem{}{\textbf{Moss, F. C.}, Rohrmeier, M. \emph{Towards a syntactic account for harmonic sequences in extended tonality}. Syntax Square Meeting, Massachusetts Institute of Technology, Department of Linguistics and Philosophy, Cambridge, USA.}
\cvitem{}{\textbf{Moss, F. C.} \emph{Syntax of Extended Tonality: Towards a Grammar of Generalized Harmonic Functions}. Music Theory Colloquium, Boston University, College of Fine Arts, School of Music, Boston, USA.}
\cvitem{}{\textbf{Moss, F. C.} \emph{Generalizing Harmonic Functions: A Grammatical Approach to Extended Tonality}. Yale University, Department of Music, New Haven, USA.}
\cvitem{}{\textbf{Moss, F. C.} \& Harasim, D. \emph{Extended Tonality and Music Cognition}. Symposium ``Towards a World Music Theory'', University of Hamburg, Institute for Systematic Musicology, Hamburg, Germany.}
\cvitem{}{\textbf{Moss, F. C.} \emph{Music Cognition and Extended Tonality: Theoretical Challenges and Empirical Implications}. Research Colloquium, University of Cologne, Cologne, Germany.}

% 2015
\cvitem{2015}{\textbf{Moss, F. C.} \emph{On generative modelling of musical form}. Seminar ``Mathematics and Music'', TUD, Dresden, Germany.}
\cvitem{}{\textbf{Moss, F. C.} \emph{`The terror of sanctity.' Tonal cues for resolving dramatic ambiguities in Wagner's Parsifal}. Seminar ``Understanding Musical Structures'', TUD, Dresden Germany.}

% 2014

\cvitem{2014}{\textbf{Moss, F. C.} \emph{Tonality and functional equivalence: A multi-level model for the cognition of triadic progressions in 19th century music}. International conference of Students of Systematic Musicology, Goldsmiths University, London, UK.}
\cvitem{}{\textbf{Moss, F. C.} \emph{Language, music and the brain: a resource-sharing framework (Patel, 2012)}. Seminar ``Cognitive Neuroscience of Music'', Institut for Musicology, University of Cologne, Cologne, Germany.}

%----------------------------------------------------------------------------------------
%	AWARDS EXPERIENCE SECTION
%----------------------------------------------------------------------------------------

\section{Teaching}
\cvitem{Fall 2020}{``Introduction to Musical Corpus Studies'', UzK.}
\cvitem{Spring 2020}{``Musical improvisation, invention and creativity'', teaching assistant, EPFL.}
\cvitem{}{``Musical Diversity across Historical Time'', lecture in class ``Digital Musicology'', EPFL.}
\cvitem{Spring 2018}{``Digital Musicology'', teaching assistant, EPFL.}
\cvitem{2016--2017}{``Reading Class Musicology'', with Christoph Wald, TUD.}
\cvitem{2015--2016}{``Introduction to Musicology'', with Christoph Wald, TUD.}
\cvitem{Spring 2013}{``Academic Writing and Research Techniques'', HfMT.}

\section{Supervision}
\cvitem{Spring 2020}{Co-supervision of Digital Humanities MSc thesis.}
\cvitem{Fall 2019}{Supervision of Machine Learning graduate student project, EPFL.}
\cvitem{Fall 2018}{Supervision of three Machine Learning graduate student projects, EPFL.}
\cvitem{Spring 2018}{Supervision of four Digital Musicology graduate student projects, EPFL.}
\cvitem{Spring 2017}{Peer-mentoring visiting PhD student in music theory/composition, TUD.}
\cvitem{Fall 2015}{Joint supervision of interdisciplinary project of technical design undergraduate, TUD.}

\section{Funding}

\subsection{Project grants}

\cventry{2021}{Collaborative Research on Science and Society (CROSS)}{
    \newline Digitizing the Dualism Debate: A Case Study in the Computational Analysis of Historical Music Theory Sources}{}{}{
    PIs: Fabian C. Moss, François Bavaud (Université de Lausanne), CHF 59'565}

\subsection{Awards and scholarships}

\cvitem{nominated}{EPFL Doctorate Award}
\cvitem{2016--2017}{Konrad Adenauer Foundation, PhD Scholarship.}
\cvitem{Aug 2016}{TUD Graduate Academy, Travel Award.}
\cvitem{Jan--Mar 2016}{Deutscher Akademischer Austauschdienst (DAAD), 
    great!\textsubscript{ipid4all} (group2group exchange for academic talents).}
\cvitem{Sep 2014}{Society for Education and Music Psychology (SEMPRE), Travel Award.}
\cvitem{Jan--Apr 2012}{European Union (EU), ERASMUS Scholarship.}
\cvitem{2008--2013}{Konrad Adenauer Foundation, Student Scholarship.}

%----------------------------------------------------------------------------------------
%	ADMIN SECTION
%----------------------------------------------------------------------------------------

\section{Administration}
\subsection{Organization}
\cvitem{2019}{Workshop ``Schenkerian and Tonfeld Theory for Music Analysis'', EPFL.}
\cvitem{}{First Swiss Digital Humanities Exchange, in collaboration with University of Basel.}
\cvitem{2015}{Co-organization of lecture series ``Systematic Musicology: Perception and Cognition of Music'', TUD.}
\cvitem{2013}{Co-organization of the international conference ``Musical Meter in Comparative Perspective'', HfMT.}

\subsection{Reviewer activity}
\cvitem{}{
    Music and Science,
    Empirical Musicology Review,
    Music Theory and Analysis,
    % \emph{Transactions of the International Society of Music Information Retrieval},
    Zeitschrift der Gesellschaft für Musiktheorie [Journal of the German-speaking Society of Music Theory],
    International Conference of Students of Systematic Musicology.
    % GMTH Proceedings
    }

\subsection{Responsibilities and memberships}
\cvitem{since 2020}{Co-Chair of the Music Analysis Interest Group of the \emph{Music Encoding Initiative} (MEI).}
\cvitem{since 2019}{UNIL-EPFL Centre for Digital Humanities (dhCenter); EPFL Data Champions Community;
    Gesellschaft f{\"u}r Musikforschung (GfM).}
\cvitem{2018--2019}{Co-founder and vice-president of the Digital Humanities Student Association \emph{dhelta} at EPFL.}
\cvitem{since 2017}{Gesellschaft f{\"u}r Musiktheorie (GMTH).}
\cvitem{2012--2013}{Financial officer for General Students' Commitee [Finanzreferent AStA], HfMT.}

%----------------------------------------------------------------------------------------
%	MEDIA SECTION
%----------------------------------------------------------------------------------------

\section{Media coverage}

\cvitem{Aug 2020}{``Bringing computational music analysis beyond the traditional canon'' \newline\scriptsize\url{https://actu.epfl.ch/news/bringing-computational-music-analysis-beyond-the-t/}}
\cvitem{Jun 2019}{``A Data Science Analysis Finds Beethoven's Style In His String Quartets'' \newline\scriptsize\url{https://www.forbes.com/sites/evaamsen/2019/06/06/a-data-science-analysis-finds-beethovens-style-in-his-string-quartets/}}
\cvitem{}{``Decoding Beethoven's music style using data science'' \newline\scriptsize\url{https://actu.epfl.ch/news/decoding-beethoven-s-music-style-using-data-scienc/}}
\cvitem{Mar 2019}{``Creating connections in a growing digital humanities community''\newline\scriptsize\url{https://actu.epfl.ch/news/creating-connections-in-a-growing-digital-humani-2/}}

%----------------------------------------------------------------------------------------
%	COURSES / SUMMER SCHOOLS SECTION
%----------------------------------------------------------------------------------------

\section{Workshops and summer schools attended}

% \subsection{Workshops and Summer Schools}
% \cvitem{2020}{Workshop ``Transcribing -- Encoding -- Annotating: New Approaches of Technology and Methodology for Historical Sources in Crowd Sourcing and Citizen Science''. Forschungsbibliothek Gotha der Universität Erfurt, November 26--27 [online].}
\cvitem{2020}{Workshop ``Musikalische Schrift und Digitalität'', Basel, Switzerland, September 22--23.}
\cvitem{}{\emph{edirom} summer school (text and music encoding). Universtity of Paderborn. Paderborn, Germany, August 31--September 4 [online].}
\cvitem{2019}{Workshop ``Research Data Management: introduction'', EPFL Library, October 10.}
\cvitem{2018}{Workshop ``Voice-leading schemata in theory, corpus research, and practical composition (Compose your own Chopin!)'', EPFL, September 18--20.}
\cvitem{}{Symposium ``Archiving Intangible Cultural Heritage \& Performing Arts: A Symposium and Summer School for Living Traditions'', EPFL, August 6--7.}
\cvitem{2017}{Workshop ``Meaning in Music: Bridging Musicological, Linguistic, and Neuroscientific Perspectives'', EPFL, December 4--6.}
\cvitem{}{Summer School ``Exploring Edges: An International Colloquium between the Digital Humanities, Architecture, Artistic Research, and Critical Technical Practice'', EPFL, July 11--14.}
\cvitem{2016}{Summer School ``Cognitive Neuroscience of Music'', University of Helsinki, August 11--17.}

% \subsection{University Courses}
% \cvitem{2017}{``Applied Data Analysis'' (Robert West), EPFL.}
% \cvitem{2016}{
% ``Introduction to Schenkerian Theory'' (Oliver Schwab-Felisch), TUD;
% ``Cognitive Science'' (P. Sinha, J. Tenenbaum, E. Gibson), MIT;
% ``Computational Modeling of Phonology and Morphology'' (T. O'Donnell, A. Albright), MIT.
% }
% \cvitem{2015}{
% ``Generative Modeling'' (T. O'Donnell), TUD;
% ``Introduction to Quantitative Methods for the Social Sciences'' (Bernhard Schipp), TUD.
% }
% \cvitem{2012-13}{``Cognitive Neuroscience of Music'', ``Cognitive Musicology: Theoretical Foundations'', ``Cognitive Modeling'' (Uwe Seifert), UzK.}


%----------------------------------------------------------------------------------------
%	COMPUTER SKILLS SECTION
%----------------------------------------------------------------------------------------

% \section{Skills}

% \cvitem{Languages}{
%     Python, Latex, HTML/CSS \newline
%     German (native), English (fluent), French (conversational), Spanish (basic)}
% \cvitem{Utilities}{Git, GitHub, Jupyter Notebook/Lab}

% \section{Musical activities}

% \cvdoubleitem{2014--2017}{Classical vocal octet \emph{Vokalexkursion}}{2008--2013}{Pop a-capella group \emph{gezwungenerma\ss en}}
% \cvdoubleitem{2013--2015}{Cologne Cathedral Chamber Choir}{since 1994}{Guitar}
% \cvdoubleitem{2011--2013}{Cologne Conservatory Chamber Choir}{since 1993}{Piano}


%----------------------------------------------------------------------------------------
%	INTERESTS SECTION
%----------------------------------------------------------------------------------------

% \section{Interests}

% \renewcommand{\listitemsymbol}{-~} % Changes the symbol used for lists

% \cvlistdoubleitem{Cycling}{Hiking}
% \cvlistdoubleitem{Sketching}{Gaming}
% \cvlistitem{Quizzing}

%----------------------------------------------------------------------------------------

% \section{References}

% \begin{multicols}{2}
% \cventry{}{K.S Suresh}{\newline Assistant Professor}{\newline Metallurgical and Materials Engg., IIT Roorkee}{\newline suresfmt@iitr.ac.in}{}
% \columnbreak
% \cventry{}{Anu Chandra}{\newline CEO}{\newline Ryelore AI}{\newline anu@ryelore.com}{ }
% \end{multicols}
% \cventry{}{Arpit Gupta}{\newline VP Engineering}{\newline Antriex IT Services}{\newline arpit.gupta@antmex.com}{}

% \cventry{}{B.S.S Daniel}{\newline Professor}{\newline Metallurgical and Materials Engg., IIT Roorkee}{\newline s4danfmt@iitr.ac.in}{ }


%% REFS
% \nocite{*}
% \printbibliography

\end{document}
