%% 

%%%%%%%%%%%%%%%%%%%%%%%%%%%%%%%%%%%%%%%%%
% "ModernCV" CV and Cover Letter
% LaTeX Template
% Version 1.3 (29/10/16)
%
% This template has been downloaded from:
% http://www.LaTeXTemplates.com
%
% Xavier Danaux (xdanaux@gmail.com) with modifications by:
% Original author:
% Vel (vel@latextemplates.com)
%
% License:
% CC BY-NC-SA 3.0 (http://creativecommons.org/licenses/by-nc-sa/3.0/)
%
% Important note:
% This template requires the moderncv.cls and .sty files to be in the same
% directory as this .tex file. These files provide the resume style and themes
% used for structuring the document.
%
%%%%%%%%%%%%%%%%%%%%%%%%%%%%%%%%%%%%%%%%%

%----------------------------------------------------------------------------------------
%	PACKAGES AND OTHER DOCUMENT CONFIGURATIONS
%----------------------------------------------------------------------------------------

\documentclass[10pt,a4paper,roman]{moderncv} % Font sizes: 10, 11, or 12; paper sizes: a4paper, letterpaper, a5paper, legalpaper, executivepaper or landscape; font families: sans or roman
% \nopagenumbers{} % suppress page numbers

\moderncvstyle{banking} % CV theme - options include: 'casual' (default), 'classic', 'oldstyle' and 'banking'
\moderncvcolor{blue} % CV color - options include: 'blue' (default), 'orange', 'green', 'red', 'purple', 'grey' and 'black'
\usepackage{multicol}
\usepackage{lipsum} % Used for inserting dummy 'Lorem ipsum' text into the template
\usepackage[hyperref]{}
\usepackage[scale=.85]{geometry} % Reduce document margins % scale=0.88
\setlength{\hintscolumnwidth}{2.2cm} % Uncomment to change the width of the dates column
% \setlength{\makecvtitlenamewidth}{10cm} % For the 'classic' style, uncomment to adjust the width of the space allocated to your name

\usepackage{hanging}
\usepackage{lmodern}
\usepackage{amsmath}

% \usepackage[
%     backend=biber,
%     style=apa,
%     natbib=true,
%     sorting=ydnt
% ]{biblatex}
% \addbibresource{bibliography.bib}

% %% make my name bold
% {
% \renewcommand*{\mkbibnamegiven}[1]{%
%   \ifitemannotation{me}
%     {\textbf{#1}}
%     {#1}}
% \renewcommand*{\mkbibnamefamily}[1]{%
%   \ifitemannotation{me}
%     {\textbf{#1}}
%     {#1}
% }

% \newcommand{\OA}{\textbf{*}}
\newcommand{\inv}{\textcolor{color2}{$\mathbf{^i}$}}
\newcommand{\OA}{\textcolor{color2}{\textbf{*}}}

%----------------------------------------------------------------------------------------
%	NAME AND CONTACT INFORMATION SECTION
%----------------------------------------------------------------------------------------

\firstname{Fabian C.} % Your first name
\familyname{Moss} % Your last name

% All information in this block is optional, comment out any lines you don't need
\title{Curriculum Vitae}
\address{Chemin de Renens 18}{CH-1004 Lausanne}
\mobile{(+41) 78 700 8485}
% \phone{(000) 111 1112}
%\fax{(000) 111 1113}
\email{fabian.moss@epfl.ch}
% \email{asinha@mt.iitr.ac.in}
\homepage{fabian-moss.de}{fabian-moss.de} % The first argument is the url for the clickable link, the second argument is the url displayed in the template - this allows special characters to be displayed such as the tilde in this example
\extrainfo{\href{https://twitter.com/fabianmoss}{{@}fabianmoss}}
% \photo[70pt][0.4pt]{pictures/picture} % The first bracket is the picture height, the second is the thickness of the frame around the picture (0pt for no frame)
% \quote{"A witty and playful quotation" - John Smith}

%%% For publictations etc.
% indentation solution from: https://tex.stackexchange.com/questions/478286/class-moderncv-cvitem-indent-from-second-line
\newcommand*{\mycvitem}[3][.25em]{%
    \begin{description}
        \item[\ifthenelse{\equal{#2}{}}{}{\hintstyle{#2}: }] #3
    \end{description}%
    \par\addvspace{#1}}

%----------------------------------------------------------------------------------------

\begin{document}

%----------------------------------------------------------------------------------------
%	COVER LETTER
%----------------------------------------------------------------------------------------

% To remove the cover letter, comment out this entire block

% \recipient{HR Department}{Corporation\\123 Pleasant Lane\\12345 City, State} % Letter recipient
% \date{\today} % Letter date
% \opening{Dear Sir or Madam,} % Opening greeting
% \closing{Sincerely yours,} % Closing phrase
% \enclosure[Attached]{curriculum vit\ae{}} % List of enclosed documents

% \makelettertitle % Print letter title

% \lipsum[1-2] % Dummy text
% \lipsum[4] % Dummy text

% \makeletterclosing % Print letter signature

% \newpage

%----------------------------------------------------------------------------------------
%	CURRICULUM VITAE
%----------------------------------------------------------------------------------------

\makecvtitle % Print the CV title

%----------------------------------------------------------------------------------------
%	EMPLOYMENT SECTION
%----------------------------------------------------------------------------------------

\section{Employment}

\cventry{from 2022}{Cultural Analytics, Data Science Center, Amsterdam, The Netherlands}{University of Amsterdam (UvA)}{Research Fellow}{}{}  % Arguments not required can be left empty
\cventry{2020--2021}{Digital and Cognitive Musicology Lab (DCML), Lausanne, Switzerland}{{\'E}cole Polytechnique F{\'e}d{\'e}rale de Lausanne (EPFL)}{Postdoctoral Researcher}{}{}  % Arguments not required can be left empty
\cventry{2017--2019}{Digital and Cognitive Musicology Lab (DCML), Lausanne, Switzerland}{{\'E}cole Polytechnique F{\'e}d{\'e}rale de Lausanne (EPFL)}{Doctoral Assistant}{}{}  % Arguments not required can be left empty
\cventry{2015--2017}{Dresden Music Cognition Lab (DMCL), Dresden, Germany}{Technische Universit{\"a}t Dresden (TUD)}{Doctoral Assistant}{}{}  % Arguments not required can be left empty
\cventry{2012--2014}{Project ``Jedem Kind seine Stimme'' (JEKISS), Leverkusen, Germany}{Musikschule Leverkusen}{Conductor and vocal coach}{}{}  % Arguments not required can be left empty
%----------------------------------------------------------------------------------------
%	EDUCATION SECTION
%----------------------------------------------------------------------------------------

\section{Education} % \& qualifications

\cventry{2017/09--2019/12}{Digital and Cognitive Musicology Lab (DCML), Lausanne, Switzerland}{{\'E}cole Polytechnique F{\'e}d{\'e}rale de Lausanne (EPFL)}{PhD student}{}{}  % Arguments not required can be left empty
\cventry{2016/01--2016/03}{Department of Linguistics and Philosophy, Cambridge, MA, USA}{Massachusetts Institute of Technology (MIT)}{Visiting Student}{}{}%}
\cventry{2015/01--2017/08}{Dresden Music Cognition Lab (DMCL), Dresden, Germany}{Technische Universit{\"a}t Dresden (TUD)}{PhD student}{}{}  % Arguments not required can be left empty
\cventry{2012/01--2012/04}{Barcelona, Spain}{Escola Superior de Musica de Catalunya (ESMUC)}{ERASMUS Exchange Student}{}{}
\cventry{2011/04--2013/09}{Musicology, Cologne, Germany}{Hochschule f{\"u}r Musik und Tanz K{\"o}ln (HfMT)}{Master of Arts}{}{}  % Arguments not required can be left empty
\cventry{2008/04--2013/09}{Music Education (Piano Major), Cologne, Germany}{Hochschule f{\"u}r Musik und Tanz K{\"o}ln (HfMT)}{Staatexamen [State Examination]}{}{}  % Arguments not required can be left empty
\cventry{2006/10--2016/09}{Mathematics and Educational Sciences, Cologne, Germany}{Universit{\"a}t zu K{\"o}ln (UzK)}{Staatexamen [State Examination]}{}{}  % Arguments not required can be left empty
\cventry{2002/09--2005/06}{Cologne, Germany}{Friedrich-Wilhelm-Gymnasium K{\"o}ln (FWG)}{Abitur [German High School Diploma]}{}{}  % Arguments not required can be left empty
% \cventry{1996--2002}{Europaschule K{\"o}ln, Gesamtschule Zollstock}{}{}{}{Cologne, Germany}
% \cventry{1992--1996}{Katholische Grundschule Trierer Stra\ss e}{}{}{}{Cologne, Germany}
%----------------------------------------------------------------------------------------
%	PUBLICATIONS SECTION
%----------------------------------------------------------------------------------------

\section{Publications \textcolor{black}{\small(\OA = Open Access)}}

\subsection{Theses}

\mycvitem{PhD}{\OA\textbf{Moss, F. C.} (2019). \emph{Transitions of Tonality: A Model-Based Corpus Study}. Doctoral dissertation. École Polytechnique Fédérale de Lausanne, Lausanne, Switzerland.
    Supervisors: Martin Rohrmeier \& Markus Neuwirth. \url{https://doi.org/10.5075/epfl-thesis-9808}}

\mycvitem{MA}{\textbf{Moss, F. C.} (2012). \emph{``Theorie der Tonfelder'' nach Simon und ``Neo-Riemannian Theory'':
    Systematik, historische Bez{\"u}ge und analytische Praxis im Vergleich}. Supervisor: Hans Neuhoff. \url{https://doi.org/10.5281/zenodo.4748512}}

\subsection{Journal Articles}

% In Preparation
% \mycvitem{in prep.}{\textbf{Moss, F. C.}, Hentschel, J., Neuwirth, M., \& Rohrmeier, M. (in prep.). The harmonic vocabulary of 19th-century piano composers.}
% \mycvitem{}{\textbf{Moss, F. C.}, Noll, T., \& Rohrmeier, M. (in prep.). Surfing the Chromatic Waves: Detecting Tone Fields Using Discrete Fourier Analysis.}
% \mycvitem{}{\textbf{Moss, F. C.}, Herff, S., \& Rohrmeier, M. (in prep.). Individual perception of diatonic scales predicts perceived tonal fit in extended tonality.} %  \emph{Psychonomic Bulletin \& Review}.
% \mycvitem{}{\textbf{Moss, F. C.}, Lieck, R., \& Rohrmeier, M. Tracing Historical Changes in Tonality with the Tonal Diffusion Model.}
% \mycvitem{}{\textbf{Moss, F. C.} Polytonality and the Emergence of Tone Fields in Tailleferre's \emph{Pastorale} (1919).}
% \mycvitem{}{Métrailler, C., Bavaud, F., \& \textbf{Moss, F. C. } CROSS-Methods}
% \mycvitem{}{Köster, M., Bavaud, F., \& \textbf{Moss, F. C. } CROSS-Close reading}
% ISMIR: Modes
% FMA: Lelegesan
% DLfM: Phantom Curves

%% Submitted

%% In Review
% \mycvitem{}{\textbf{Moss, F. C.}, Affatato, G., \& Harasim, D. (2022). Phantom Curves: Scientific Discovery through Interactive Music Visualisation. \emph{9th International Conference on Digital Libraries for Musicology}. 28--29 July, 2022, Prague.}
% \mycvitem{}{Meng, S., \textbf{Moss, F. C.}, \& Rohrmeier, M. (in review). What makes a Gong -- A study in Chinese pentatonic scale. \emph{7th Biennial International Conference on Analytical Approaches to World Musics (AAWM 2022)}. University of Sheffield, Sheffield, UK, June 14--17, 2022.}
% \mycvitem{}{\OA Herff, S. A., \textbf{Moss, F. C.}, \& Rohrmeier, M. (2021, October 15). Evidence for cognitive tonal hierarchies in cadential but not scalar contexts. Retrieved from \url{osf.io/yz957} [Preprint].}

%% Accepted

%% In Press

%% 2022
\mycvitem{2022}{\OA \textbf{Moss, F. C.}, Neuwirth, M., \& Rohrmeier, M. (2022). The line of fifths and the co-evolution of tonal pitch-classes. \emph{Journal of Mathematics and Music}. \url{https://doi.org/10.1080/17459737.2022.2044927}}
\mycvitem{}{\OA Viaccoz, C., Harasim, D., \textbf{Moss, F. C.}, \& Rohrmeier, M. (2022). Wavescapes: A Visual Hierarchical Analysis of Tonality Using the Discrete Fourier Transformation. \emph{Musicae Scientiae}. \url{https://doi.org/10.1177/10298649211034906}}

%% 2021
\mycvitem{2021}{\OA \textbf{Moss, F. C.} \& Rohrmeier, M. (2021). Discovering tonal structures with Latent Dirichlet Allocation. \emph{Music \& Science}, 4(20592043211048827). \url{https://doi.org/10.1177/20592043211048827}}
\mycvitem{}{\OA \textbf{Moss, F. C.} \& Neuwirth, M. (2021). FAIR, Open, Linked: Introducing the Special Issue on Open Science in Musicology. \emph{Empirical Musicology Review} 16(1), 1--4. \url{http://dx.doi.org/10.18061/emr.v16i1.8246}}
\mycvitem{}{\OA Harasim, D., \textbf{Moss, F. C.}, Ramirez, M., \& Rohrmeier, M. (2021). Exploring the foundations of tonality: Statistical cognitive modeling of modes in the history of Western classical music. \emph{Humanities and Social Sciences Communications}, 8(5), 1--11. \url{https://doi.org/10.1057/s41599-020-00678-6}}

%% 2020
\mycvitem{2020}{\OA Lieck, R., \textbf{Moss, F. C.}, \& Rohrmeier, M. (2020). The Tonal Diffusion Model. \emph{Transactions of the Inter\-national Society of Music Information Retrieval}, 3(1), 153--164. \url{https://doi.org/10.5334/tismir.46}}
\mycvitem{}{\OA \textbf{Moss, F. C.}, de Souza, W. F., \& Rohrmeier, M. (2020). Harmony and Form in Brazilian Choro: A Corpus-Driven Approach to Musical Style Analysis. \emph{Journal of New Music Research},  49(5), 416--437. \url{https://doi.org/10.1080/09298215.2020.1797109}}

%% 2019
\mycvitem{2019}{\OA \textbf{Moss, F. C.}, Neuwirth, M., Harasim, D., \& Rohrmeier, M. (2019). Statistical characteristics of tonal harmony: A corpus study of Beethoven's string quartets. \emph{PLOS ONE}, 14(6), e0217242. \url{https://doi.org/10.1371/journal.pone.0217242}}
\mycvitem{}{\OA Popescu, T., Neuser, M. P., Neuwirth, M., Bravo, F., Mende, W., Boneh, O., \textbf{Moss, F. C.}, \& Rohrmeier, M. (2019). The pleasantness of sensory dissonance is mediated by musical style and expertise. \emph{Scientific Reports}, 9(1), 1070. \url{https://doi.org/10.1038/s41598-018-35873-8}}

%% 2018
\mycvitem{2018}{\OA Neuwirth, M., Harasim, D., \textbf{Moss, F. C.}, \& Rohrmeier, M. (2018). The Annotated Beethoven Corpus~(ABC): A Dataset of Harmonic Analyses of All Beethoven String Quartets. \emph{Frontiers in Digital Humanities}, 5(July), 1--5. \url{https://doi.org/10.3389/fdigh.2018.00016}}

%% Earlier
\mycvitem{2017}{\textbf{Moss, F. C.} (2017). [Review of David Huron. Voice Leading: The Science behind a Musical Art]. Music Theory \& Analysis, 4(1), 119--130. \url{https://doi.org/10.11116/MTA.4.1.7}}
\mycvitem{2012}{\OA \textbf{Moss, F. C.} \emph{Albert Simons \emph{Theorie der Tonfelder} und John Cloughs \emph{Flip-Flop Circles} im Vergleich}. Zenodo. \url{http://doi.org/10.5281/zenodo.3944462}}

\subsection{Conference Proceedings}

\mycvitem{accepted}{Harasim, D., Affatato, G., \& \textbf{Moss, F. C.} (2022). midiVERTO: A Web Application to Visualize Tonality in Real Time. \emph{8th International Conference on Mathematics and Computation in Music (MCM2022)}. Georgia State University, Atlanta, USA, 21--24 June 2022.}
\mycvitem{2021}{\OA \textbf{Moss, F. C.}, Köster, M., Femminis, M., Metrailler, C., \& Bavaud, F. (2021). Digitizing a 19th-century music theory debate for computational analysis. In: In M. Ehrmann, F. Karsdorp, M. Wevers, T. L. Andrews, M. Burghardt, M. Kestemont, E. Manjavacas, M. Piotrowski, \& J. van Zundert (Eds.), \emph{CHR 2021: Computational Humanities Research Conference}, 
    November 17–19, 2021, Amsterdam, The Netherlands (pp. 159–170). \url{http://ceur-ws.org/Vol-2989/short_paper31.pdf}}
\mycvitem{}{\OA Rohrmeier, M., \& \textbf{Moss, F. C.} (2021). A Formal Model of Extended Tonal Harmony
In: \emph{Proceedings of the 22nd International Society for Music Information Retrieval Conference.} [Online]. \url{https://archives.ismir.net/ismir2021/paper/000071.pdf}}
\mycvitem{}{\OA Hentschel, J., \textbf{Moss, F. C.}, \& Rohrmeier, M. (2021). A Semi-Automated Workflow Paradigm for the Distributed
Creation and Curation of Expert Annotations. In: \emph{Proceedings of the 22nd International Society for Music Information Retrieval Conference.} [Online]. \url{https://archives.ismir.net/ismir2021/paper/000032.pdf}}
\mycvitem{}{\OA Hentschel, J., \textbf{Moss, F. C.}, McLeod, A., Neuwirth, M., \& Rohrmeier, M. (2021). Towards a Unified Model of Chords in Western Harmony. In: \emph{Music Encoding Conference Proceedings 2021}. Alicante, Spain [Online].}
\mycvitem{}{\OA Anzuoni, E., Ayhan, S., Dutto, F., Mcleod, A., \textbf{Moss, F. C.}, \& Rohrmeier, M. (2021). A Historical Analysis of Harmonic Progressions Using Chord Embeddings. \emph{Proceedings of the 18th Sound and Music Computing Conference}, 284–291. \url{https://doi.org/10.5281/zenodo.5038910}}
\mycvitem{2019}{\OA Landnes, K., Mehrabyan, L., Wiklund, V., Lieck, R., \textbf{Moss, F. C.}, \& Rohrmeier, M. (2019). A
    Model Comparison for Chord Prediction on the Annotated Beethoven Corpus. In I.
    Barbancho, L. J. Tard{\'o}n, A. Peinado, \& A. M. Barbancho (Eds.), \emph{Proceedings of the 16th Sound and Music Computing Conference (SMC 2019)} (pp. 250--254). M{\'a}laga, Spain.}
\mycvitem{}{\OA \textbf{Moss, F. C.} (2014). Tonality and functional equivalence: A multi-level model for the cognition of triadic progressions in 19th century music. In \emph{International Conference of Students of Systematic Musicology -- Proceedings} (pp. 1--8). London, UK.}

\subsection{Preprints}

\mycvitem{2022}{\OA Harasim, D., Affatato, G., \& \textbf{Moss, F. C.} (2022, March 25). midiVERTO: A Web Application to Visualize Tonality in Real Time. \emph{arXiv:2203.13158 [cs]}. \url{https://arxiv.org/abs/2203.13158}}
\mycvitem{2021}{\OA Herff, S. A., \textbf{Moss, F. C.}, \& Rohrmeier, M. (2021, October 15). Evidence for cognitive tonal hierarchies in cadential but not scalar contexts. \url{https://osf.io/yz957}}

\subsection{Book chapters}

\mycvitem{in prep.}{\textbf{Moss, F. C.}. Transatlantic transformations: How Riemannian is Neo-Riemannian theory?
In S. Keym \& C. Hust (Eds.), \emph{Hugo Riemann: Musikforschung zwischen Universalität, Nationalismus und internationaler Ausstrahlung}. Leipzig.}

\subsection{As editor}

\mycvitem{2021}{\textbf{Moss, F. C.} \& Neuwirth, M. (Eds.) (2021). \emph{Empirical Musicology Review} 16(1) [Special Issue ``Open Science in Musicology''].}

\subsection{Datasets and Code}

\mycvitem{2022}{\OA \textbf{Moss, F. C.}, Harasim, D., \& Affatato, G. (2022). midiVERTO: a web application for interactive music visualization using the DFT. \url{https://github.com/DCMLab/midiVERTO}}
\mycvitem{2020}{\OA \textbf{Moss, F. C.}, Neuwirth, M., \& Rohrmeier, M. (2020). Tonal Pitch-Class Counts Corpus (TP3C) [Data set]. \emph{Zenodo}. \url{https://doi.org/10.5281/zenodo.3600080}}
\mycvitem{2019}{\OA \textbf{Moss, F. C.}, Loayza, T., \& Rohrmeier, M. (2019). pitchplots. \emph{Zenodo}. \url{https://doi.org/10.5281/zenodo.3265392}}
\mycvitem{2018}{\OA \textbf{Moss, F. C.}, de Souza, W. F., \& Rohrmeier, M. (2018). Choro Songbook Corpus [Data set]. \emph{Zenodo}. \url{https://doi.org/10.5281/zenodo.1442764}}

\subsection{Blogposts}

\mycvitem{2020}{``A computational model for note distributions in musical pieces''\newline \url{https://www.epfl.ch/labs/dcml/computational-model-note-dists/}}
\mycvitem{}{``Tracing historical changes in the exploration of tonal space''\newline \url{https://www.epfl.ch/labs/dcml/tracing-historical-changes/}}
%----------------------------------------------------------------------------------------
%	TALKS SECTION
%----------------------------------------------------------------------------------------

\section{Outreach} % \small(\,=\,~invited)

\begin{multicols}{2}

\subsection{Invited talks and workshops}

%% 2025
  \mycvitem{2025}{\textbf{Moss, F. C.} \& Lesemann-Elliott, C. \emph{Data collecting from musical sources}. EarlyMuse COST Action 21161, Working Group 2 ('Sources'). 
  Institute of Musicology of the Slovak Academy of Sciences \& University Library, 17--19 March, 2025, Bratislava, Slovakia.}

%% 2024
\mycvitem{2024}{\textbf{Moss, F. C.} \emph{Computational Musicology: oxymoron or perfect fit?} Graduate Schools Day, Julius-Maximilians-Universität Würzburg, 4 July 2024, Würzburg, Germany.}
\mycvitem{}{\textbf{Moss, F. C.} \emph{Künstliche Intelligenz und Musikwissenschaft}. Austrian Centre for Digital Humanities and Cultural Heritage (ACDH-CH) of the Austrian Academy of Sciences (ÖAW), Vienna, Austria, 17 Juni 2024.}
\mycvitem{}{\textbf{Moss, F. C.} \emph{Corpus Research and Choro: Potential and Challenges for Digital Methods}. 3 April 2024. School of Music, Universidade Federal do Rio de Janeiro, Rio de Janeiro, Brazil.}

%% 2023
\mycvitem{2023}{\textbf{Moss, F. C.} \emph{Virtual Tonal Spaces (VTS): towards an interactive digital environment for music theory}. Tag der Lehre 2023. Julius-Maximilians-Universität Würzburg, 22 November 2023, Würzburg, Germany.}
\mycvitem{}{\textbf{Moss, F. C.} \emph{10 secret rules for a degree in DH -- you won't believe no. 7!!}. 1st DH Alumni Event, 17 November 2023. Digital Humanities Insitute, École Polytechnique Fédérale de Lausanne, Switzerland.}
\mycvitem{}{\textbf{Moss, F. C.} \emph{Counting notes: Research questions and methods in music corpus studies}. Seminar ``History and Theory of Digital Humanities'', Université de Lausanne, Lausanne, Switzerland, 12 October 2023.}
\mycvitem{}{\textbf{Moss, F. C.} \emph{Töne zählen: Forschungsfragen und Methoden musikwissenschaftlicher Korpusstudien in historischer und epistemologischer Perspektive}. Talk in lecture series ``Transdisziplinäre Aspekte digitaler Methodik in den Geistes- und Kulturwissenschaften'', Leibniz-Institut für Europäische Geschichte, Mainz, Germany, 28 June, 2023.}
\mycvitem{}{\textbf{Moss, F. C.} \emph{Musik Er-Zählen: Einblicke in die digitale Korpusforschung.} Vortrag im Institutskolloquium des Instituts für Musikforschung, JMU, 20 June, 2023.}{}
\mycvitem{}{\textbf{Moss, F. C.} Respondent to Sanja Kiš Žuvela: \emph{Musical Terminology, Digital Corpus Management and Translation}. GMTH International Music Theory Lectures. 9 February, 2023.}
\mycvitem{}{Arthur, C., Baker, D., Burgoyne, J. A., Cecchetti, G., Eerola, T., Farbood, M., Finkensiep, C., Harrison, P., Koelsch, S., Margulis, E., \textbf{Moss, F. C.}, Neuwirth, M., Pearce, M., Pelofi, C., Rammos, Y., Rohrmeier, M., \& Volk, A. \emph{Decoding Musical Structure: Theory, Computation, and Neuroscience} (Workshop). Congressi Stefano Franscini, Monte Verità, 5--9 February, 2023.}

%% 2022
\mycvitem{2022}{ \textbf{Moss, F. C.} \emph{Music Stylometry -- the Case of Choro}. Music Cognition Lab Meeting, Princeton University [online], 2 November, 2022.}
\mycvitem{}{\textbf{Moss, F. C.} \emph{Learning about Machine Learning with CRIM}. Digital Counterpoints: Exploring Similarity in Renaissance Music, October 20--22, 2022, Haverford College, Department of Music, Haverford, PA.}
\mycvitem{}{ \textbf{Moss, F. C.} \emph{midiVERTO: A web-based tool to make computational music analysis more accessible}. Institute für Musik und Musikwissenschaft, Technische Universität Dortmund, Germany, 28 April.}
\mycvitem{}{ \textbf{Moss, F. C.} \emph{Interactive Music Analysis using the DFT and Pitch-Class Distributions extracted from MIDI files}. Faculdade de Engenharia da Universidade do Porto (FEUP), Porto, Portugal, 4 April 2022.}
\mycvitem{}{ \textbf{Moss, F. C.} \emph{Music Theory and the Discrete Fourier Transform}. Cognitive and Systematic Musicology Lab Meeting, The Ohio State University, Columbus, USA [online], 25 March 2022.}

%% 2021
\mycvitem{2021}{ \textbf{Moss, F. C.} \emph{The Science of Music.} EPFL Information Days, 24--25 November 2021, Lausanne, Switzerland. \url{https://youtu.be/y5TQN09zDVI}}
\mycvitem{}{\textbf{Moss, F. C.} \emph{Boosting Open Research in Empirical Musicology}. EPFL Data Champions Meeting (DCBreak\#3). March 18, 2021, Lausanne, Switzerland [online].}

%% 2020
\mycvitem{2020}{\textbf{Moss, F. C.} \emph{The Importance of Modeling in Computational Musicology}. Round-table on ``Probability and Music'', 5th International Congress of Music and Mathematics (MusMat 2020) -- Perspectives and Applications of Mathematics in Post-Tonal Theories («Homage to Jamary Oliveira»),
    December 8--12, Rio de Janeiro, Brazil [online].}
    \mycvitem{}{\textbf{Moss, F. C.} \emph{Data-Driven Music History}. Workshop for the International Conference of Students of Systematic Musicology, York University, September 14, 2020, York, UK [online].}
\mycvitem{}{\textbf{Moss, F. C.} \emph{Computational Musicology and the Digital Humanities: Problems, Practices, and Prospects}. CRETA-Werkstatt \#9, Center for Reflected Text Analytics, University of Stuttgart, February 18, 2020, Stuttgart, Germany.}

%% 2019
\mycvitem{2019}{\textbf{Moss, F. C.} \emph{Tracing the History of Tonality with Note Distributions}. ``Corpus Research as a Means of Unlocking Musical Grammar'' International Research Workshop, July 1--4, 2019, Tel-Aviv, Israel.}

%% 2018
\mycvitem{2018}{\textbf{Moss, F. C.} \emph{Corpus Research in Digital Musicology}. Seminar ``Willkommen in der Matrix: Digitale Anwendungen f{\"u}r die Musikanalyse in Theorie und Praxis'', University of Basel, Basel, Switzerland.}

%% 2017
\mycvitem{2017}{\textbf{Moss, F. C.} \emph{Formal Grammars and Ambiguity in Extended Tonality}. Workshop and Symposium on Schenkerian Analysis ``Wege der Kreativit{\"a}t -- Zwischen Erfindung und Rekonstruktion'', Universit{\"a}t der K{\"u}nste, Berlin, Germany.}
\mycvitem{}{\textbf{Moss, F. C.} \emph{From Beethoven to Brazil: Digital Musicology at EPFL}. Digital Synergies: Ca' Foscari meets École Polytechnique Fédérale de Lausanne. Global Challenges Seminar - Team ``Creative arts, cultural heritage and digital humanities'', Venice, Italy.}

%% 2016
\mycvitem{2016}{\textbf{Moss, F. C.} \emph{Extended Tonality: Theoretical Challenges and their Relation to the Neuroscientific Study of Musical Syntax}. Max Planck Institute for Human Cognitive and Brain Sciences, Leipzig, Germany. }
\mycvitem{}{\textbf{Moss, F. C.}, Rohrmeier, M. \emph{Towards a syntactic account for harmonic sequences in extended tonality}. Syntax Square Meeting, Massachusetts Institute of Technology, Department of Linguistics and Philosophy, Cambridge, USA.}
\mycvitem{}{\textbf{Moss, F. C.} \& Harasim, D. \emph{Extended Tonality and Music Cognition}. Symposium ``Towards a World Music Theory'', University of Hamburg, Institute for Systematic Musicology, Hamburg, Germany.}

\subsection{Conference presentations}

%% 2025 
%\mycvitem{2025}{Roeder, T., Klinger, J., Stickler, F., Keupp, C., \& \textbf{Moss, F. C.} \emph{Multimodality and Minimal Publishing: TEI, MEI and more in 19th-Century Music Treatises}. 25th Annual Meeting of the Text Encoding Initiative. 16--17 September 2025, Kraków, Poland.}
\mycvitem{2025}{\textbf{Moss, F. C.} \emph{Vuza-Kanons: ein mathematisches Konzept zur Beschreibung nicht-überlappender Rhythmen}. XXIV. Kontrapunktwerkstatt. 8--10 May, 2025, Basel, Switzerland.}
\mycvitem{}{Nachtwey, A., \& \textbf{Moss, F. C.} \emph{Beyond Bars: Distribution of Differences in Music Prints}. Music Encoding Conference 2025. 3--6 June 2025, London, UK.}
\mycvitem{}{Stickler, F., Roeder, T., \& \textbf{Moss, F. C.} \emph{A Minimal Publishing Model for Text and Music Notation}. Music Encoding Conference 2025. 3--6 June 2025, London, UK.}

%% 2024
\mycvitem{2024}{\textbf{Moss, F. C.} \& Nakamura, E. \emph{Modeling the evolution of harmony in popular music from different cultural contexts}. CHR2024: Fifth Conference on Computational Humanities Research, 4--6 December 2024, Aarhus, Denmark.}
\mycvitem{}{Hofmann, L., Sapp, C. S., \& \textbf{Moss, F. C.} \emph{Metrical Irregularities and Polymetric Structures in Hugo Distler's Vocal Works: Towards a Digital Corpus Study.} 2nd International Conference on Computational and Cognitive Musicology, 17--18 October 2024, Utrecht, The Netherlands.}
\mycvitem{}{Nachtwey, A. \& \textbf{Moss, F. C.} \emph{Digitale Korpusbildung in der Musikforschung: Herausforderungen und Lösungsansätze für die quantitative Analyse von Musikeditionsvarianten.} Jahrestagung der Gesellschaft für Musikforschung 2024, 11--14 September 2024, Cologne, Germany.}
\mycvitem{}{Hofmann, L. \& \textbf{Moss, F. C.} \emph{``Zeitgemäß polyphon''. Zur Kodierung und Modellierung von Polymetrik und metrischer Irregularität in Hugo Distlers Vokalwerken.} Jahrestagung der Gesellschaft für Musikforschung, 11--14 September 2024, Cologne, Germany.}
\mycvitem{}{Eipert, T. \& \textbf{Moss, F. C.} \emph{Digital Paths Through History: Phylogenetic Analysis of Medieval Chants from the Graduale Synopticum Data.} [Poster] Jahrestagung der Gesellschaft für Musikforschung, 11--14 September 2024, Cologne, Germany.}
\mycvitem{}{\textbf{Moss, F. C.} \& Nakamura, E. \emph{Cross-cultural modeling of the evolution of harmony in popular music}. Cultural Evolution Society Conference (CES 24), 9--11 September 2024, Durham, UK.}
\mycvitem{}{Eipert, T., \textbf{Moss, F. C.}, \& Vlhóva-Wörner, H. \emph{Reconstructing the Formation of Trope Traditions through Network Models}. Annual International Medieval and Renaissance Music Conference (MedRen) 2024, 6--9 July, 2024, Granada, Spain.}
\mycvitem{}{Polykarpidis, P., Kalofonos, Dionysios., \textbf{Moss, F. C.}, \& Anagnostopoulou, C. \emph{Echos (mode) classification in heirmologic corpora of Byzantine music}. [Poster] Annual International Medieval and Renaissance Music Conference (MedRen) 2024, 6--9 July, 2024, Granada, Spain.}
\mycvitem{}{Hofmann, T., Sapp, C., \& \textbf{Moss, F. C.} \emph{Encoding polymeters and metric irregularities in selected motets from Hugo Distler's \emph{Der Jahrkreis} op. 5 using different music encoding formats.} ECHOES conference ``Digital Technologies Applied to Music Research: Methodologies, Projects and Challenges'', 27-29 June 2024, Lisbon, Portugal.}
\mycvitem{}{Eipert, T., Hartelt, A., \textbf{Moss, F. C.}, Puppe, F. \emph{Medieval Chant Lineages Unlocked: Leveraging Optical Music Recognition for Phylogenetic Analysis of Gregorian Proper.} ECHOES conference ``Digital Technologies Applied to Music Research: Methodologies, Projects and Challenges'', 27-29 June 2024, Lisbon, Portugal.}
\mycvitem{}{Pereira, S., Affatato, G., Bernardes, G., \& \textbf{Moss, F. C.} \emph{Fourier Qualia Wavescapes: Hierarchical Analyses of Set Class Quality and Ambiguity}. 9th International Conference on Mathematics and Computation in Music (MCM2024). Universidade de Coimbra, Coimbra, Portugal, 18--21 June 2024.}

% 2023
\mycvitem{2023}{Nakamura, E., Eipert, T. \& \textbf{Moss, F. C.} \emph{Historical Changes of Modes and their Substructure Modeled as Pitch Distributions in Plainchant from the 1100s to the 1500s}. 16th International Symposium on Computer Music Multidisciplinary Resarch (CMMR2023), 13--17 November 2023, Tokyo, Japan.}
\mycvitem{}{Eipert, T. \& \textbf{Moss, F. C.} \emph{MonodiKit: A data model and toolkit for the Corpus Monodicum}. The 10th International Conference on Digital Libraries for Musicology (DLfM '23), 10 November 2022, Milano, Italy.}{}
\mycvitem{}{Eipert, T. \& \textbf{Moss, F. C.} \emph{Communities in Medieval Troper Networks are Shaped by Carolingian Politics}. Poster. The 10th International Conference on Digital Libraries for Musicology (DLfM '23), 10 November 2022, Milano, Italy.}{}
\mycvitem{}{Yust, J., Affatato, G., \& \textbf{Moss, F. C.}. \emph{Animated Harmonic Analysis Using DFT Phase Spaces and Coefficient Products}. Joint Annual Meeting of the American Musicological Society (AMS) and the Society for Music Theory (SMT), 9--12 November 2023, Denver, Colorado.}
\mycvitem{}{\textbf{Moss, F. C.} \emph{Korpusforschung und Digitale Edition: ein Plädoyer für stärkere Intradisziplinarität}. Beitrag im Panel ``Musikalische Korpusforschung: Aktuelle Trends und Herausforderungen'', mit Markus Neuwirth, Martin Rohrmeier, Christof Weiß, Johannes Hentschel \& Maik Köster. Jahrestagung der Gesellschaft für Musikforschung, 4--7 October, 2023, Saarbrücken, Germany.}
\mycvitem{}{Eipert, T., Frieler, K., \& \textbf{Moss, F. C.} \emph{Inside or Outside: The Use of Scales in Jazz Solo Improvisations}. Poster.Jahrestagung der Gesellschaft für Musikforschung, 4--7 October, 2023, Saarbrücken, Germany.}
\mycvitem{}{\textbf{Moss, F. C.} [Cancelled.] \emph{Star Plots: eine neue Methode zur Visualisierung harmonischer Pfade für vierstimmige Kompositionen}. 23. Jahreskongress der Gesellschaft für Musiktheorie (GMTH) ``Musiktheorie und Künstlerische Forschung'', 22-24 September 2023, Hochschule für Musik Freiburg, Freiburg im Breisgau, Germany.}
\mycvitem{}{Roeder, T., Köster, M., \& \textbf{Moss, F. C.} \emph{Music-Text Interlinking as a Challenge for Digital Encodings of Music-Theoretical Writings}. Encoding Cultures -- Joint MEC and TEI Conference 2023, 4--8 September 2023, Zentrum Musik -- Edition -- Medien (ZenMEM), Paderborn, Germany.}
\mycvitem{}{Eipert, T. \& \textbf{Moss, F. C.} \emph{A system of trope elements: using network models to understand interrelations within the transmission of trope complexes}. Annual International Medieval and Renaissance Music Conference (MedRen) 2023, 24--28 July, 2023, Munich, Germany.}

% 2022
\mycvitem{2022}{\textbf{Moss, F. C.}, Nápoles López, N., Köster, M. \&  Rizo, D. \emph{Challenging sources: a new dataset for OMR of diverse 19th-century music theory examples}. 4th International Workshop on Reading Music Systems (WoRMS 2022), 18 November 2022 [online].}
\mycvitem{}{Köster, M. \& \textbf{Moss, F. C.} \emph{Der harmonische Dualismus und seine Entwicklung zum `Streit- und Angelpunkt der Musiktheorie' -- eine Diskursanalyse}. Jahrestagung der Gesellschaft für Musikforschung. Nach der Norm: Musikwissenschaft im 21. Jahrhundert, 29 September -- 1 October 2022, Humboldt-Universität Berlin, Berlin, Germany.}
\mycvitem{}{\textbf{Moss, F. C.} \& Métrailler, C. [Cancelled.] \emph{Reading Music Theory from a Distance: A Corpus Study of the Thesaurus Musicarum Italicarum}. 21st Quinquennial Congress of the International Musicological Society (IMS2022), 22--26 August 2022, Athens, Greece.}
\mycvitem{}{\textbf{Moss, F. C.}, Affatato, G. \& Harasim, D. \emph{Phantom Curves: Scientific Discovery through Interactive Music Visualization}. The 9th International Conference on Digital Libraries for Musicology (DLfM), In association with the annual conference of the International Association of Music Libraries (IAML), 28 July 2022, Prague, Czech Republic.}{}
\mycvitem{}{Harasim, D., Affatato, G., \& \textbf{Moss, F. C.}. \emph{midiVERTO: A Web Application to Visualize Tonality in Real Time}. 8th International Conference on Mathematics and Computation in Music (MCM2022). Georgia State University, Atlanta, USA, 21--24 June 2022.}
\mycvitem{}{Bracks, C. \& \textbf{Moss, F. C.} \emph{Totoli's Art of Lelegesan: Analyzing Sociocultural Context and Musical Content.} 10th International Workshop on Folk Music Analysis 2022 (FMA2022), University of Sheffield, Sheffield, UK, June 14--17, 2022.}
\mycvitem{}{Meng, S., \textbf{Moss, F. C.}, \& Rohrmeier, M. \emph{Revisiting Tong Yun San Gong theory in Chinese music: a corpus study of Chinese folksongs.} 7th Analytical Approaches to World Music Conference (AAWM2022), University of Sheffield, Sheffield, UK, June 14--17, 2022.}

% 2021
\mycvitem{2021}{\textbf{Moss, F. C.}, Köster, M., Femminis, M., Métrailler, C., \& Bavaud, F. \emph{Digitizing a 19th-century music theory debate for computational analysis}. CHR 2021: Computational Humanities Research Conference,
November 17--19, 2021, Amsterdam, The Netherlands [online].}
\mycvitem{}{\textbf{Moss, F. C.} \emph{Polytonality and the Emergence of Tone Fields in Tailleferre's \emph{Pastorale}}. 21. Jahreskongress der Gesellschaft für Musiktheorie (GMTH) -- Tonsysteme und Stimmungen.
    October 1--3, 2021, Musik-Akademie Basel/Hochschule für Musik (FHNW), Basel, Switzerland.}
\mycvitem{}{Hentschel, J., \textbf{Moss, F. C.}, Markus Neuwirth, \& Rohrmeier, M. \emph{Die Entwicklung der tonalen Sprache in Beethovens Streichquartetten: Eine vergleichende Korpusstudie der Schaffensphasen}. XVII. Internationaler Kongress der Gesellschaft für Musikforschung, Universität Bonn, Abteilung für Musikwissenschaft/Sound Studies und Beethoven-Archiv des Beethoven-Hauses Bonn
Bonn, Germany, September 28 -- October 1 2021, Bonn, Germany.}
\mycvitem{}{\textbf{Moss, F. C.} \emph{Digitizing the Dualism Debate: a case study in the computational analysis of historical music theory sources}. CROSS 2021 Event. 16 September 2021, École Polytechnique Fédérale de Lausanne/Université de Lausanne, Lausanne, Switzerland.}
\mycvitem{}{\textbf{Moss, F. C.}, Herff, S. A., \& Rohrmeier, M. \emph{Modeling perceived tonal stability of individual and aggregated listener responses for scales and cadences}. 16th International Conference on Music Perception and Cognition \& 11th triennial conference of the European Society for the Cognitive Sciences of Music. July 28--31, Sheffield, UK [online].}
\mycvitem{}{\textbf{Moss, F. C.}, Herff, S. A., \& Rohrmeier, M. \emph{Individual perception of diatonic scales predicts perceived tonal fit in octatonic and hexatonic contexts}. 16th International Conference on Music Perception and Cognition \& 11th triennial conference of the European Society for the Cognitive Sciences of Music. July 28--31, Sheffield, UK [online].}
\mycvitem{}{Hentschel, J., \textbf{Moss, F. C.}, McLeod, A., \& Rohrmeier, M. \emph{Towards a Unified Model of Chords in Western Harmony}. Music Encoding Conference [online].}
\mycvitem{}{Anzuoni, E., Ayhan, S., Dutto, F., McLeod, A., \textbf{Moss, F. C.}, \& Rohrmeier, M. \emph{A Historical Analysis of Harmonic Progressions Using Chord Embeddings}. 18th Sound and Music Computing Conference [online].}
\mycvitem{}{\textbf{Moss, F. C.} \emph{Discovering the line of fifths in a large historical corpus}. Future Directions of Music Cognition, The Ohio State University, March 6--7, 2021, Columbus, OH [online]. \url{https://doi.org/10.17605/OSF.IO/J5W6T}}

%% 2020
\mycvitem{2020}{\textbf{Moss, F. C.} \emph{Analyzing musical pieces on the Tonnetz using the \emph{pitchplots} Python library}. 20. Jahreskongress der Gesellschaft für Musiktheorie
    (GMTH), Hochschule für Musik Detmold, October 1--4, 2020, Detmold, Germany [online].}
% 2019
\mycvitem{2019}{\textbf{Moss, F. C.} \emph{Transitions of Tonality: Perspectives on the Historical Changes of Tonal Pitch Relations from Computational Musicology, Music Theory, and the Digital Humanities}. University of Cologne, November 29, 2019, Cologne, Germany.}
\mycvitem{}{\textbf{Moss, F. C.} \emph{Inferring Tonality from Note Distributions -- Why Models Matter (Poster)}. SEMPRE Graduate Conference 2019, Cambridge, UK.}
\mycvitem{}{\textbf{Moss, F. C.} \emph{Analyzing Tonality with Note Distributions}. First Swiss Digital Humanities Student Exchange DHX2019, Basel, Switzerland. }

% 2018
\mycvitem{2018}{\textbf{Moss, F. C.}, Souza, W. F. \& Rohrmeier, M. \emph{Harmony and Form in Brazilian Choro: A Corpus Study}. 15th International Conference on Music Perception and Cognition \& 10th triennial conference of the European Society for the Cognitive Sciences of Music, Graz, Austria. }
\mycvitem{}{Aitken, C., O'Donnell, T. \& Rohrmeier, M. [Poster presented by \textbf{Moss, F. C.}]. \emph{A Maximum Likelihood Model for the Harmonic Analysis of Symbolic Music}. 15th Sound and Music Computing Conference ``Sonic Crossings''. Limassol, Cyprus.}
\mycvitem{}{Harasim, D., \textbf{Moss, F. C.} \& Ramirez, M. \emph{A Brief History of Tonality} (Poster). Applied Machine Learning Days, EPFL, Switzerland.}

% 2017
\mycvitem{2017}{\textbf{Moss, F. C.}, Souza, W. F. \& Rohrmeier, M. \emph{Brazilian Choro: A New Data Set of Chord Transcriptions and Analyses of Harmonic and Formal Features}. 17. Jahreskongress der Gesellschaft f{\"u}r Musiktheorie (GMTH) \& 27. Arbeitstagung der Gesellschaft f{\"u}r Popularmusikforschung (GfPM) ``Popul{\"a}re Musik und ihre Theorien: Begegnungen -- Perspektivwechsel -- Transfers'', Graz, Austria. }
\mycvitem{}{\textbf{Moss, F. C.}, Harasim, D., Neuwirth, M. \& Rohrmeier, M. \emph{Beethovens Streichquartette -- ein XML-basierter Korpus harmonischer Analysen in einem neuen Annotationssystem}. Jahrestagung der Gesellschaft f{\"u}r Musikforschung, Kassel, Germany.}
\mycvitem{}{\textbf{Moss, F. C.}, Rohrmeier, M. \emph{Integrating Transformational and Hierarchical Models of Extended Tonality}. 9th European Music Analysis Conference (EuroMAC), Strasbourg, France.}
\mycvitem{}{Rom, U., Je\ss ulat, A., \textbf{Moss, F. C.} \& Guter, I. \emph{Ambiguity, Illusion \& Timelessness in Late and Post-Tonal Harmony}. Panel discussion at the 9th European Music Analysis Conference (EuroMAC), Strasbourg, France.}
\mycvitem{}{\textbf{Moss, F. C.}, Rohrmeier, M. \& Bravo, F. \emph{Emotional Associations Evoked by Structural Properties of Musical Scales and Abstract Visual Shapes}. KOSMOS Dialogue ``Music, Emotion, and Visual Imagery'', Berlin, Germany.}
\mycvitem{}{Harasim, D., \textbf{Moss, F. C.}, Neuwirth, M. \& Rohrmeier M. \emph{Beethoven's String Quartets: Introducing an XML-Based Corpus of Harmonic Labels Using a New Annotation System}. Music Encoding Conference, Tours, France.}

% 2016
\mycvitem{2016}{\textbf{Moss, F. C.}, Rohrmeier, M. \emph{Structural Ambiguities in Language and Music} (Poster). Helsinki Summer School for Cognitive Neuroscience 2016 (HSSCN 2016).}
\mycvitem{}{\textbf{Moss, F. C.}, Rohrmeier, M.  \emph{A grammatical approach to tension-resolution patterns in extended tonal harmony}. Meeting of the Computational Cognitive Science Group, Massachusetts Institute of Technology, Department of Brain and Cognitive Sciences, Cambridge, USA.}
\mycvitem{}{\textbf{Moss, F. C.} \emph{Syntax of Extended Tonality: Towards a Grammar of Generalized Harmonic Functions}. Music Theory Colloquium, Boston University, College of Fine Arts, School of Music, Boston, USA.}
\mycvitem{}{\textbf{Moss, F. C.} \emph{Generalizing Harmonic Functions: A Grammatical Approach to Extended Tonality}. Yale University, Department of Music, New Haven, USA.}
\mycvitem{}{\textbf{Moss, F. C.} \emph{Music Cognition and Extended Tonality: Theoretical Challenges and Empirical Implications}. Research Colloquium, University of Cologne, Cologne, Germany.}

% 2015
\mycvitem{2015}{\textbf{Moss, F. C.} \emph{On generative modelling of musical form}. Seminar ``Mathematics and Music'', TUD, Dresden, Germany.}
\mycvitem{}{\textbf{Moss, F. C.} \emph{`The terror of sanctity.' Tonal cues for resolving dramatic ambiguities in Wagner's Parsifal}. Seminar ``Understanding Musical Structures'', TUD, Dresden Germany.}

% 2014

\mycvitem{2014}{\textbf{Moss, F. C.} \emph{Tonality and functional equivalence: A multi-level model for the cognition of triadic progressions in 19th century music}. International conference of Students of Systematic Musicology, Goldsmiths University, London, UK.}
% \mycvitem{}{\textbf{Moss, F. C.} \emph{Language, music and the brain: a resource-sharing framework (Patel, 2012)}. Seminar ``Cognitive Neuroscience of Music'', Institut for Musicology, University of Cologne, Cologne, Germany.}

\subsection{Science Communication}
\mycvitem{2024}{\textbf{Moss, F. C.}, \emph{Musik und Mathematik -- Wie passt das eigentlich zusammen?}, Science Slam 2024, Julius-Maximilians-Universität Würzburg, 8 November 2024. \url{https://www.uni-wuerzburg.de/alumni/alle-veranstaltungen/science-slam/science-slam-2024/}}
\mycvitem{2021}{ Rohrmeier, M. \& \textbf{Moss, F. C.} \emph{Music, Mathematics, and the Geometry of Jazz}. Montreux Jazz Festival, July 11, 2021, Montreux, Switzerland. \url{https://www.montreuxjazzfestival.com/de/artist/martin-rohrmeier/}}
\mycvitem{2017}{\textbf{Moss, F. C.} \emph{Musik und Sprache}. Talk for Student Association ``Denkzettel'', TUD, Dresden, Germany.}

\end{multicols}

\section{Teaching and mentoring}

\subsection{Courses}

\mycvitem{Fall 2023}{``CODAMUS: Computational and Digital Approaches to Music Scholarship'' (international lecture series);
``Die Entstehung von `Tonalität' im 19. Jahrhundert'', JMU}
\mycvitem{Spring 2023}{``Musikalische Korpusforschung''; 
    ``Konzepte und Anwendungen der Pitch-Class Set Theory''; 
    ``Digitale Tools (nicht nur) für Musikwissenschaftliche Projektarbeiten'', JMU}
\mycvitem{Fall 2022}{``Neo-Riemannian Theories: Analysemethoden für erweiterte Tonalität von der Spätromantik bis zur Filmmusik'';
    ``Music Memes: Quantitative Zugänge und Theorien zu kultureller Transmission von Musik'', JMU}
\mycvitem{Spring 2021}{``Musical Diversity across Historical Time'', lecture in class ``Digital Musicology'', EPFL}
\mycvitem{Fall 2020}{``Introduction to Musical Corpus Studies''; 
    ``Tonality: Perspectices of historical musicology and corpus studies'', lecture in ``Ringvorlesung Musikwissenschaft'', UzK}
\mycvitem{Spring 2020}{``Musical improvisation, invention and creativity'', teaching assistant;
    ``Musical Diversity across Historical Time'', lecture in class ``Digital Musicology'', EPFL}
\mycvitem{Spring 2018}{``Digital Musicology'', teaching assistant, EPFL}
\mycvitem{2015--2017}{``Introduction to Musicology'' and ``Reading Class Musicology'', with Christoph Wald, TUD}
\mycvitem{Spring 2013}{``Academic Writing and Research Techniques'', HfMT}

\subsection{PhD thesis supervision}
% \cvitem{04/2024--today}{Supervision of Adrian Nachtwey: 
%     ``Eine Studie zur textkritischen Analyse von Musikeditionsvarianten 
%     im 19. Jahrhundert unter Anwendung von digitalen Methoden'' 
%     (Musicology), JMU}
\cvitem{10/2023--today}{Supervision of Tim Eipert \& Lucas Hofmann, JMU}
\cvitem{07/2022--today}{Co-supervision of Shuxin Meng (with Martin Rohrmeier), Digital Humanities, EPFL}
\cvitem{Spring 2017}{Peer-mentoring visiting PhD student Willian Fernandes de Souza:
    ``Estilo e Sintaxe: quatro ensaios analíticos em práticas do choro'', Music Theory/Composition,
    Universidade Federal do Rio de Janeiro (UFRJ)}

\subsection{Master thesis supervision}
\cvitem{Spring 2023}{%
Supervision of Julia Groblewski-Meiser: ``Narration und Interpretation: 
    Allegorische Darstellungen einer musikalischen Harmonie 
    im Kuppelfresko von Santa Maria del Fiore von Giorgio Vasari'',
    Musicology, JMU;
    co-supervision of Oscar Aquite Pena (with Nepomuk Riva), 
    Ethnomusicology, JMU; 
}
\cvitem{Spring 2020}{Co-supervision of Cédric Viaccoz: ``Visual Hierarchical Analysis of Tonality using the
Discrete Fourier Transform'', Digital Humanities (with Daniel Harasim \& Martin Rohrmeier), EPFL}

\subsection{Bachelor thesis supervsion}
\cvitem{Spring 2023}{Supervision of Corinna Bongartz: 
``Musik und Künstliche Intelligenz: Eine Untersuchung der Zuordnung 
festgelegter Prompts zu durch Sprachmodellen erzeugt Musiksnippets'', 
Musicology, JMU}
\cvitem{Spring 2022}{Co-supervision of Iris Folpmers: ``Data Sonification: Turning Climate Data into
Music'', Artificial Intelligence (with Tobias Blanke), UvA, \url{https://scripties.uba.uva.nl/search?id=record_29490}}

\subsection{Other mentoring}
\cvitem{Fall 2020}{3 Machine Learning graduate student projects on vector embeddings of harmony (EPFL)}
\cvitem{Fall 2019}{Machine Learning graduate student project on vector embeddings of harmony (EPFL)}
\cvitem{Fall 2018}{3 Machine Learning graduate student projects on chord prediction with neural networks (EPFL)}
\cvitem{Spring 2018}{4 Digital Musicology graduate student projects (EPFL)}
\cvitem{Fall 2015}{interdisciplinary project of technical design undergraduate, Technische Universität Dresden (TUD)}
\section{Supervision and mentoring}

\subsection{PhD thesis supervision}
% \mycvitem{2024--today}{Silas Bischoff: ``Aufstieg und Fall der Deutschen Lautentabulatur -- Eine Untersuchung zu ihrem Ursprung und zu ihrer Entwicklungsgeschichte'' (Musicology), JMU}
\mycvitem{10/2023--today}{Adrian Nachtwey: 
    ``Eine Studie zur textkritischen Analyse von Musikeditionsvarianten 
    im 19. Jahrhundert unter Anwendung von digitalen Methoden'' (Musicology), JMU}
\mycvitem{}{Tim Eipert: ``A Quantitative Perspective on Transmission, Structure, and Modality of Medieval Chant'', Graduate School Humanities (Digital Humanities), JMU}
\mycvitem{}{Lucas Hofmann: ``Computational modeling of complex temporal and tonal structures in early twentieth-century music'', Graduate School Humanities (Digital Humanities), JMU}
\mycvitem{07/2022--today}{Shuxin Meng, Digital Humanities, EPFL (1st supervisor: Martin Rohrmeier)}
\mycvitem{Spring 2017}{Willian Fernandes de Souza (peer-mentoring):
    ``Estilo e Sintaxe: quatro ensaios analíticos em práticas do choro'' (Music Theory/Composition), Universidade Federal do Rio de Janeiro (UFRJ)}

\subsection{Master thesis supervision}
\mycvitem{Fall 2023}{%
    Francesco Paolo Leonardo La Barbera: ``Proportionen, Transformationen oder Tonfelder? Die vergleichende Anwendung dreier musiktheoretischer Ansätze'' (Musicology), Universität Leipzig (1st supervisor: Stefan Keym)}
\mycvitem{}{Felicitas Stickler: ``Das Passionsoratorium „Der sterbende Heiland“ von Ignaz Franz Xaver Kürzinger. Edition -- Kritischer Bericht -- Analytische Aspekte'' (Musicology), JMU (1st supervisor: Ulrich Konrad)}
\mycvitem{}{Julia Groblewski-Meiser: ``Narration und Interpretation:  Allegorische Darstellungen einer musikalischen Harmonie im Kuppelfresko von Santa Maria del Fiore von Giorgio Vasari'' (Musicology), JMU}
\mycvitem{}{Oscar Aquite Pena: ``Between millo and picó: music as discursive masking in
 \emph{La Puntica No Ma'}, costume troupe of the Barranquilla Carnival (Colombia)'' (Ethnomusicology), JMU (1st supervisor: Nepomuk Riva)}
\cvitem{Spring 2020}{Cédric Viaccoz (Digital Humanities, 3rd supervisor): ``Visual Hierarchical Analysis of Tonality using the
Discrete Fourier Transform'', EPFL}

\subsection{Bachelor thesis supervsion}
\cvitem{Fall 2024}{Felicitas Stickler (Digital Humanities): ``Ein minimales Modell für die gemischte Kodierung von Text (TEI) und Musiknotation (MEI)'', Digital Humanities, JMU}
\cvitem{Spring 2023}{Corinna Bongartz (Musicology): 
``Musik und Künstliche Intelligenz: Eine Untersuchung der Zuordnung 
festgelegter Prompts zu durch Sprachmodellen erzeugt Musiksnippets'', 
Musicology, JMU}
\cvitem{Spring 2022}{Iris Folpmers (Artificial Intelligence, 2nd supervisor): ``Data Sonification: Turning Climate Data into
Music'' (Artificial Intelligence), UvA, \url{https://scripties.uba.uva.nl/search?id=record_29490}}

\subsection{Other mentoring}
\cvitem{Fall 2024}{Digital-Humanities Projekt ``Digitale Präsentation von XML-kodierten musiktheoretischen Texten mit CETEIcean'' (Felicitas Stickler)}
\cvitem{Fall 2020}{3 Machine Learning graduate student projects on vector embeddings of harmony (EPFL)}
\cvitem{Fall 2019}{Machine Learning graduate student project on vector embeddings of harmony (EPFL)}
\cvitem{Fall 2018}{3 Machine Learning graduate student projects on chord prediction with neural networks (EPFL)}
\cvitem{Spring 2018}{4 Digital Musicology graduate student projects (EPFL)}
\cvitem{Fall 2015}{interdisciplinary project of technical design undergraduate, Technische Universität Dresden (TUD)}

\section{Funding}

\subsection{Grants}


%\cventry{2025/01--2028/12}{Musico Pratico: Pontio}{Schweizerischer Nationalfonds}{city}{grade}{}
\cventry{2025/05}{RISM Digital Center}{EU COST Action \emph{EarlyMuse} Short-Term Scientific Mission (STSM)}{EUR 2,000}{}{Fabian C. Moss}
\cventry{2025/01/01--202512/31}{%
	\parbox{.8\textwidth}{Aufbau einer offenen digitalen Sammlung historischer
musiktheoretischer Texte aus dem deutschsprachigen Raum \\%
anhand von Beispielen aus dem 19. Jahrhundert (DigiMusTh)}}{Text+ Kooperationsprojekt}{EUR 74,652}{}{Fabian C. Moss}
\cventry{2024/01--2024/08}{Digital Choro: Exploring the potential of digitization and computational models 
    for Brazil's musical cultural heritage}{%
    Bayerisches Hochschulförderprogramm zur Anbahnung \& Vertiefung internat. Forschungskooperationen}{%
    EUR 3,863}{}{Fabian C. Moss}

\cventry{2023/10--2024/06}{Virtual tonal spaces (VTS): towards an interactive digital environment for music theory
}{WueDIVE -- Digitale Innovationen in der Lehre}{EUR 3,915}{}{Fabian C. Moss}

\cventry{2022/12--2023/11}{%
    Start-up funding to prepare grant application [Anschubförderung zur Antragsstellung]
    }{Julius-Maximilians-Universität Würzburg}{EUR 8,310}{}{Fabian C. Moss}

\cventry{2022/06--2023/05}{%
Funding for proof-of-concept study to support larger grant application
}{Durham University Seedcorn Grant
}{GBP 7,475}{}{Tuomas Eerola, Fabian C. Moss}

\cventry{2022/04--2024/12}{%
Data Scientists/Engineers Cultural Data Access \& Visualization, Spatial Humanities, Cultural Data Analysis
}{University of Amsterdam Data Science Centre Accelerate Program (Matching Funding)}{EUR 192,000}{}{
    Tobias Blanke, Fabian C. Moss, Julia Noordegraaf, \& Thomas Poell}

\cventry{2021/07--2021/09}{%
Enabling interactive music visualization for a wider community
}{dhCenter UNIL-EPFL project fund}{CHF 2'880}{}{
    Fabian C. Moss \& Daniel Harasim}

\cventry{2021/01--2021/12}{%
    Digitizing the Dualism Debate: A Case Study in the Computational Analysis of Historical Music Theory Sources
}{Collaborative Research on Science and Society (CROSS)}{CHF 59'565}{}{
    Fabian C. Moss \& François Bavaud}

\subsection{Awards and scholarships}

\textbf{2016--2017: } Konrad Adenauer Foundation, PhD Scholarship;
\textbf{2016/08: } TUD Graduate Academy, Travel Award;
\textbf{2016/01--03} Deutscher Akademischer Austauschdienst (DAAD),
    great!\textsubscript{ipid4all} (group2group exchange for academic talents);
\textbf{2014/09: } Society for Education and Music Psychology (SEMPRE), Travel Award;
\textbf{2012/01--04: } European Union (EU), ERASMUS Scholarship;
\textbf{2008--2013: } Konrad Adenauer Foundation, Student Scholarship

\section{Administration}

\subsection{Organization}

% \mycvitem{2022}{Workshop ``Representing Harmony: Goals and Challenges'', with Johannes Hentschel. XX--XX March 2022, Digital and Cognitive Musicology Lab, École Polytechnique Fédérale de Lausanne, Switzerland.}
\mycvitem{2021}{Workshop ``Musik -- Schrift -- Digitalität'' [Music -- Writing -- Digitality], with Dennis Ried and Daniel Fütterer. 13--14 December 2021, Hochschule für Musik, Karlsruhe, Germany.}
\mycvitem{2019}{Workshop ``Schenkerian and Tonfeld Theory for Music Analysis''. 12--15 December 2019, Digital and Cognitive Musicology Lab, École Polytechnique Fédérale de Lausanne, Switzerland.}
\mycvitem{}{First Swiss Digital Humanities Exchange, with Jessica Pidoux, Gerhad Lauer and Stefan Münnich. 8--9 February 2019, DH Lab, University of Basel, Switzerland.}
\mycvitem{2015}{Co-organization of lecture series ``Systematic Musicology: Perception and Cognition of Music'', lead: Martin Rohrmeier. Dresden Music Cognition Lab, Technichal University Dresden, Germany.}
\mycvitem{2013}{Co-organization of the international conference ``Musical Metre in Comparative Perspective'', lead: Hans Neuhoff and Rainer Polak. 4--6 April 2013, Hochschule für Musik und Tanz Köln, Germany.}

\subsection{Reviewer activity}
\mycvitem{Journals}{
    \emph{Empirical Musicology Review},
    \emph{Music and Science},
    \emph{Music Theory and Analysis},
    \emph{Transactions of the International Society of Music Information Retrieval},
    \emph{Zeitschrift der Gesellschaft für Musiktheorie % [Journal of the German-speaking Society of Music Theory]
    }
}

\mycvitem{Conferences}{
    \emph{Computational Humanities Research~(CHR)},
    \emph{Conference of the European Society for the Cognitive Sciences of Music~(ESCOM)},
    \emph{International Conference on Music Perception and Cognition~(ICMPC)},
    \emph{International Conference of Students of Systematic Musicology (SysMus)}
}

\subsection{Responsibilities and memberships}
\mycvitem{since 2021}{International Society for Music Information Retrieval}
\mycvitem{since 2020}{Co-Chair of the Music Analysis Interest Group of the \emph{Music Encoding Initiative} (MEI).}
\mycvitem{since 2019}{UNIL-EPFL Centre for Digital Humanities (dhCenter); EPFL Data Champions Community;
    Gesellschaft f{\"u}r Musikforschung (GfM).}
\mycvitem{2018--2019}{Co-founder and vice-president of the Digital Humanities Student Association \emph{dhelta} at EPFL.}
\mycvitem{since 2017}{Gesellschaft f{\"u}r Musiktheorie (GMTH).}
\mycvitem{2012--2013}{Financial officer for General Students' Commitee [Finanzreferent AStA], HfMT.}

\section{Media coverage}

\cvitem{Mar 2023}{``Harmonie modellieren''\newline\scriptsize\url{https://www.uni-wuerzburg.de/aktuelles/einblick/single/news/harmonie-modellieren/}}
\cvitem{Jan 2021}{``Machine learning helps retrace evolution of classical music''\newline\scriptsize\url{https://actu.epfl.ch/news/machine-learning-helps-retrace-evolution-of-clas-2/}}
\cvitem{Aug 2020}{``Bringing computational music analysis beyond the traditional canon'' \newline\scriptsize\url{https://actu.epfl.ch/news/bringing-computational-music-analysis-beyond-the-t/}}
\cvitem{Jun 2019}{``A Data Science Analysis Finds Beethoven's Style In His String Quartets'' \newline\scriptsize\url{https://www.forbes.com/sites/evaamsen/2019/06/06/a-data-science-analysis-finds-beethovens-style-in-his-string-quartets/}}
\cvitem{}{``Decoding Beethoven's music style using data science'' \newline\scriptsize\url{https://actu.epfl.ch/news/decoding-beethoven-s-music-style-using-data-scienc/}}
\cvitem{Mar 2019}{``Creating connections in a growing digital humanities community''\newline\scriptsize\url{https://actu.epfl.ch/news/creating-connections-in-a-growing-digital-humani-2/}}
\section{Workshops and summer/winter schools attended}

\mycvitem{2020}{Workshop ``Transcribing -- Encoding -- Annotating: New Approaches of Technology and Methodology for Historical Sources in Crowd Sourcing and Citizen Science''. Forschungsbibliothek Gotha der Universität Erfurt, November 26--27 [online].}
\mycvitem{}{Workshop ``Musikalische Schrift und Digitalität'', Basel, Switzerland, September 22--23.}
\mycvitem{}{\emph{edirom} summer school (text and music encoding). Universtity of Paderborn. Paderborn, Germany, August 31--September 4 [online].}
\mycvitem{2019}{Workshop ``Research Data Management: introduction'', EPFL Library, October 10.}
\mycvitem{2018}{Workshop ``Voice-leading schemata in theory, corpus research, and practical composition (Compose your own Chopin!)'', EPFL, September 18--20.}
\mycvitem{}{Symposium ``Archiving Intangible Cultural Heritage \& Performing Arts: A Symposium and Summer School for Living Traditions'', EPFL, August 6--7.}
\mycvitem{2017}{Workshop ``Meaning in Music: Bridging Musicological, Linguistic, and Neuroscientific Perspectives'', EPFL, December 4--6.}
\mycvitem{}{Summer School ``Exploring Edges: An International Colloquium between the Digital Humanities, Architecture, Artistic Research, and Critical Technical Practice'', EPFL, July 11--14.}
\mycvitem{2016}{Summer School ``Cognitive Neuroscience of Music'', University of Helsinki, August 11--17.}

% \subsection{University Courses}
% \cvitem{2017}{``Applied Data Analysis'' (Robert West), EPFL.}
% \cvitem{2016}{
% ``Introduction to Schenkerian Theory'' (Oliver Schwab-Felisch), TUD;
% ``Cognitive Science'' (P. Sinha, J. Tenenbaum, E. Gibson), MIT;
% ``Computational Modeling of Phonology and Morphology'' (T. O'Donnell, A. Albright), MIT.
% }
% \cvitem{2015}{
% ``Generative Modeling'' (T. O'Donnell), TUD;
% ``Introduction to Quantitative Methods for the Social Sciences'' (Bernhard Schipp), TUD.
% }
% \cvitem{2012-13}{``Cognitive Neuroscience of Music'', ``Cognitive Musicology: Theoretical Foundations'', ``Cognitive Modeling'' (Uwe Seifert), UzK.}

\section{Skills}

\mycvitem{Languages}{
    Python, Latex, HTML/CSS \newline
    German (native), English (fluent), French (conversational), Spanish (basic)}
\mycvitem{Utilities}{Git, GitHub, Jupyter Notebook/Lab}

\section{Musical activities}

\cvdoubleitem{2014--2017}{Classical vocal octet \emph{Vokalexkursion}}{2008--2013}{Pop a-capella group \emph{gezwungenerma\ss en}}
\cvdoubleitem{2013--2015}{Cologne Cathedral Chamber Choir}{since 1994}{Guitar}
\cvdoubleitem{2011--2013}{Cologne Conservatory Chamber Choir}{since 1993}{Piano}


%----------------------------------------------------------------------------------------
%	INTERESTS SECTION
%----------------------------------------------------------------------------------------

% \section{Interests}

% \renewcommand{\listitemsymbol}{-~} % Changes the symbol used for lists

% \cvlistdoubleitem{Cycling}{Hiking}
% \cvlistdoubleitem{Sketching}{Gaming}
% \cvlistitem{Quizzing}

%----------------------------------------------------------------------------------------

% \section{References}

% \begin{multicols}{2}
% \cventry{}{K.S Suresh}{\newline Assistant Professor}{\newline Metallurgical and Materials Engg., IIT Roorkee}{\newline suresfmt@iitr.ac.in}{}
% \columnbreak
% \cventry{}{Anu Chandra}{\newline CEO}{\newline Ryelore AI}{\newline anu@ryelore.com}{ }
% \end{multicols}
% \cventry{}{Arpit Gupta}{\newline VP Engineering}{\newline Antriex IT Services}{\newline arpit.gupta@antmex.com}{}

% \cventry{}{B.S.S Daniel}{\newline Professor}{\newline Metallurgical and Materials Engg., IIT Roorkee}{\newline s4danfmt@iitr.ac.in}{ }


%% REFS
% \nocite{*}
% \printbibliography

\end{document}
