\section{Organization}

% \mycvitem{2024}{Themed session ``Computational Modeling and Analysis of Medieval Chant'', with Charles Atkinson, Ashley Burgoyne, Bas Cornelissen, Tim Eipert, Jan Haji\v{c}, \& Andreas Haug. Annual International Medieval and Renaissance Music Conference (MedRen 2024), 6--9 July, 2024, Granada, Spain.}
\mycvitem{2024}{Workshop ``Bayesian Modeling for Musicology'' with Christoph Finkensiep (University of Amsterdam) and Jan Haji\v{c} (Charles University, Prague), 1--3 February, 2024. Zentrum für Philologie und Digitalität (ZPD), JMU, Würzburg, Germany. \url{https://sites.google.com/view/bayesmusic2024/}}
\mycvitem{2023}{Lecture series ``CODAMUS: Computational and Digital Approaches to Music Scholarship'', 18 October, 2023--07 February, 2024. Zentrum für Philologie und Digitalität (ZPD), JMU, Würzburg, Germany. \url{https://codamus.pubpub.org/}}
\mycvitem{}{Kontrapunkt-Werkstatt ``Latest Tools for Analyzing Early Music'', with Hansjörg Ewert, Florian Vogt, Johannes Menke, \& Ugo Bindini, 20--21 October, Würzburg, Germany. \url{https://www.musikwissenschaft.uni-wuerzburg.de/diversa/tagungen/basel23/}}
\mycvitem{}{Open project space for 16 contributions ``Methoden und Ziele digitaler Musikwissenschaft: Ein Marktplatz aktueller Forschung'', with Stefanie Acquavella-Rauch, Martin Albrecht-Hohmeier, Irmlind Capelle, Jürgen Diet, \& Jens Dufner. Jahrestagung der Gesellschaft für Musikforschung, 4--7 October, 2023, Saarbrücken, Germany.\\\url{https://www.uni-saarland.de/methoden-und-ziele-digitaler-musikwissenschaft-ein-marktplatz-aktueller-forschung.html}}
\mycvitem{2022}{CREATE Salon on ``Computational Creativity'', 23 November 2022, \emph{Creative Amsterdam: An E-Humanities Perspective}, Media Studies Department, University of Amsterdam, The Netherlands. \url{https://www.create.humanities.uva.nl/events/computational-creativity/}}
\mycvitem{}{Workshop ``Representing Harmony: Goals and Challenges'', with Johannes Hentschel, Markus Neuwirth \& Martin Rohrmeier. 13--16 September 2022, Digital and Cognitive Musicology Lab, École Polytechnique Fédérale de Lausanne, Switzerland. \url{https://www.epfl.ch/labs/dcml/workshops/representing-harmony/}}
\mycvitem{2021}{Workshop ``Musik -- Schrift -- Digitalität'' [Music -- Writing -- Digitality], with Dennis Ried and Daniel Fütterer. 13--14 December 2021, Hochschule für Musik, Karlsruhe, Germany.}
\mycvitem{2019}{Workshop ``Schenkerian and Tonfeld Theory for Music Analysis''. 12--15 December 2019, Digital and Cognitive Musicology Lab, École Polytechnique Fédérale de Lausanne, Switzerland.\\ \url{https://memento.epfl.ch/event/masterclass-schenkerian-and-tonfeld-theory-for-mus/}}
\mycvitem{}{First Swiss Digital Humanities Exchange, with Jessica Pidoux, Gerhad Lauer, and Stefan Münnich. 8--9 February 2019, DH Lab, University of Basel, Switzerland. \url{https://sites.google.com/view/dhexchange/}}
\mycvitem{2015}{Co-organization of lecture series ``Systematic Musicology: Perception and Cognition of Music'', lead: Martin Rohrmeier. Dresden Music Cognition Lab, Technichal University Dresden, Germany.}
\mycvitem{2013}{Co-organization of the international conference ``Musical Metre in Comparative Perspective'', lead: Hans Neuhoff and Rainer Polak. 4--6 April 2013, Hochschule für Musik und Tanz Köln, Germany.}
