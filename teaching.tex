\section{Teaching and mentoring}

\subsection{Courses}

\mycvitem{Fall 2023}{``CODAMUS: Computational and Digital Approaches to Music Scholarship'' (international lecture series);
``Die Entstehung von `Tonalität' im 19. Jahrhundert'', JMU}
\mycvitem{Spring 2023}{``Musikalische Korpusforschung''; 
    ``Konzepte und Anwendungen der Pitch-Class Set Theory''; 
    ``Digitale Tools (nicht nur) für Musikwissenschaftliche Projektarbeiten'', JMU}
\mycvitem{Fall 2022}{``Neo-Riemannian Theories: Analysemethoden für erweiterte Tonalität von der Spätromantik bis zur Filmmusik'';
    ``Music Memes: Quantitative Zugänge und Theorien zu kultureller Transmission von Musik'', JMU}
\mycvitem{Spring 2021}{``Musical Diversity across Historical Time'', lecture in class ``Digital Musicology'', EPFL}
\mycvitem{Fall 2020}{``Introduction to Musical Corpus Studies''; 
    ``Tonality: Perspectices of historical musicology and corpus studies'', lecture in ``Ringvorlesung Musikwissenschaft'', UzK}
\mycvitem{Spring 2020}{``Musical improvisation, invention and creativity'', teaching assistant;
    ``Musical Diversity across Historical Time'', lecture in class ``Digital Musicology'', EPFL}
\mycvitem{Spring 2018}{``Digital Musicology'', teaching assistant, EPFL}
\mycvitem{2015--2017}{``Introduction to Musicology'' and ``Reading Class Musicology'', with Christoph Wald, TUD}
\mycvitem{Spring 2013}{``Academic Writing and Research Techniques'', HfMT}

\subsection{PhD thesis supervision}
\mycvitem{10/2023--today}{Adrian Nachtwey: 
    ``Eine Studie zur textkritischen Analyse von Musikeditionsvarianten 
    im 19. Jahrhundert unter Anwendung von digitalen Methoden'' (Musicology), JMU}
\mycvitem{10/2023--today}{Tim Eipert: ``A Quantitative Perspective on Transmission, Structure, and Modality of Medieval Chant'', Graduate School Humanities (Digital Humanities), JMU}
\mycvitem{}{Lucas Hofmann: ``Computational modeling of complex temporal and tonal structures in early twentieth-century music'', Graduate School Humanities (Digital Humanities), JMU}
\mycvitem{07/2022--today}{Shuxin Meng (2nd supervisor), Digital Humanities, EPFL}
\mycvitem{Spring 2017}{Willian Fernandes de Souza (peer-mentoring):
    ``Estilo e Sintaxe: quatro ensaios analíticos em práticas do choro'' (Music Theory/Composition), Universidade Federal do Rio de Janeiro (UFRJ)}

\subsection{Master thesis supervision}
\mycvitem{Spring 2023}{%
Julia Groblewski-Meiser: ``Narration und Interpretation: 
    Allegorische Darstellungen einer musikalischen Harmonie 
    im Kuppelfresko von Santa Maria del Fiore von Giorgio Vasari'' (Musicology), JMU}
\mycvitem{}{Oscar Aquite Pena (2nd supervisor) (Ethnomusicology), JMU}
\cvitem{Spring 2020}{Cédric Viaccoz (3rd supervisor): ``Visual Hierarchical Analysis of Tonality using the
Discrete Fourier Transform'' (Digital Humanities), EPFL}

\subsection{Bachelor thesis supervsion}
\cvitem{Spring 2023}{Corinna Bongartz: 
``Musik und Künstliche Intelligenz: Eine Untersuchung der Zuordnung 
festgelegter Prompts zu durch Sprachmodellen erzeugt Musiksnippets'', 
Musicology, JMU}
\cvitem{Spring 2022}{Iris Folpmers (2nd supervisor): ``Data Sonification: Turning Climate Data into
Music'' (Artificial Intelligence), UvA, \url{https://scripties.uba.uva.nl/search?id=record_29490}}

\subsection{Other mentoring}
\cvitem{Fall 2020}{3 Machine Learning graduate student projects on vector embeddings of harmony (EPFL)}
\cvitem{Fall 2019}{Machine Learning graduate student project on vector embeddings of harmony (EPFL)}
\cvitem{Fall 2018}{3 Machine Learning graduate student projects on chord prediction with neural networks (EPFL)}
\cvitem{Spring 2018}{4 Digital Musicology graduate student projects (EPFL)}
\cvitem{Fall 2015}{interdisciplinary project of technical design undergraduate, Technische Universität Dresden (TUD)}