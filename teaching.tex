\section{Teaching}

\mycvitem{Fall 2024}{``YouTube Music Theory''; ``Programmieren für Musikforschende''; ``Wie über Musik forschen? (Institutskolloquium)''}
\mycvitem{Spring 2024}{``Metrum, Rhythmus, Takt und Beat - theoretische und psychologische Aspekte musikalischer Zeit'';
``Einführung in die Digitale Musikwissenschaft'', ``Musikforschung interdisziplinär'' (Institutskolloquium), JMU}
\mycvitem{Fall 2023}{``CODAMUS: Computational and Digital Approaches to Music Scholarship'' (international lecture series);
``Die Entstehung von `Tonalität' im 19. Jahrhundert'', JMU}
\mycvitem{Spring 2023}{``Musikalische Korpusforschung''; 
    ``Konzepte und Anwendungen der Pitch-Class Set Theory''; 
    ``Digitale Tools (nicht nur) für Musikwissenschaftliche Projektarbeiten'', JMU}
\mycvitem{Fall 2022}{``Neo-Riemannian Theories: Analysemethoden für erweiterte Tonalität von der Spätromantik bis zur Filmmusik'';
    ``Music Memes: Quantitative Zugänge und Theorien zu kultureller Transmission von Musik'', JMU}
\mycvitem{Spring 2021}{``Musical Diversity across Historical Time'', lecture in class ``Digital Musicology'', EPFL}
\mycvitem{Fall 2020}{``Introduction to Musical Corpus Studies''; 
    ``Tonality: Perspectices of historical musicology and corpus studies'', lecture in ``Ringvorlesung Musikwissenschaft'', UzK}
\mycvitem{Spring 2020}{``Musical improvisation, invention and creativity'', teaching assistant;
    ``Musical Diversity across Historical Time'', lecture in class ``Digital Musicology'', EPFL}
\mycvitem{Spring 2018}{``Digital Musicology'', teaching assistant, EPFL}
\mycvitem{2015--2017}{``Introduction to Musicology'' and ``Reading Class Musicology'', with Christoph Wald, TUD}
\mycvitem{Spring 2013}{``Academic Writing and Research Techniques'', HfMT}
