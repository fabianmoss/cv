%----------------------------------------------------------------------------------------
%	TALKS SECTION
%----------------------------------------------------------------------------------------

\section{Talks, Conference Presentations, and Workshops \small(\inv\,=\,~invited)}

\subsection{Workshops and Public Talks}

%% 2021
\mycvitem{2021}{\inv \textbf{Moss, F. C.} \emph{The Science of Music.} EPFL Information Days, 24--25 November 2021, Lausanne, Switzerland. \url{https://youtu.be/y5TQN09zDVI}}
\mycvitem{}{\inv Rohrmeier, M. \& \textbf{Moss, F. C.} \emph{Music, Mathematics, and the Geometry of Jazz}. Montreux Jazz Festival, July 11, 2021, Montreux, Switzerland.}

%% 2020
\mycvitem{2020}{\textbf{Moss, F. C.} \emph{Data-Driven Music History}. International Conference of Students of Systematic Musicology, York University, September 14, 2020, York, UK [online].}

%% 2018
\mycvitem{2018}{\inv\textbf{Moss, F. C.} \emph{Corpus Research in Digital Musicology}. Seminar ``Willkommen in der Matrix: Digitale Anwendungen f{\"u}r die Musikanalyse in Theorie und Praxis'', University of Basel, Basel, Switzerland.}

\subsection{Conference presentations}

% 2022
\mycvitem{2022}{Köster, M. \& \textbf{Moss, F. C.} \emph{Der harmonische Dualismus und seine Entwicklung zum `Streit- und Angelpunkt der Musiktheorie' -- eine Diskursanalyse}. Jahrestagung der Gesellschaft für Musikforschung. Nach der Norm: Musikwissenschaft im 21. Jahrhundert, 29 September -- 1 October 2022, Humboldt-Universität Berlin, Berlin, Germany.}
\mycvitem{2022}{\textbf{Moss, F. C.} \& Métrailler, C. \emph{Reading Music Theory from a Distance: A Corpus Study of the Thesaurus Musicarum Italicarum}. 21st Quinquennial Congress of the International Musicological Society (IMS2022), 22--26 August 2022, Athens, Greece.}

% 2021
\mycvitem{2021}{\textbf{Moss, F. C.}, Köster, M., Femminis, M., Métrailler, C., \& Bavaud, F. \emph{Digitizing a 19th-century music theory debate for computational analysis}. CHR 2021: Computational Humanities Research Conference,
November 17--19, 2021, Amsterdam, The Netherlands [online].}
\mycvitem{}{\textbf{Moss, F. C.} \emph{Polytonality and the Emergence of Tone Fields in Tailleferre’s \emph{Pastorale}}. 21. Jahreskongress der Gesellschaft für Musiktheorie (GMTH) -- Tonsysteme und Stimmungen.
    October 1--3, 2021, Musik-Akademie Basel/Hochschule für Musik (FHNW), Basel, Switzerland.}
\mycvitem{}{Hentschel, J., \textbf{Moss, F. C.}, Markus Neuwirth, \& Rohrmeier, M. \emph{Die Entwicklung der tonalen Sprache in Beethovens Streichquartetten: Eine vergleichende Korpusstudie der Schaffensphasen}. XVII. Internationaler Kongress der Gesellschaft für Musikforschung, Universität Bonn, Abteilung für Musikwissenschaft/Sound Studies und Beethoven-Archiv des Beethoven-Hauses Bonn
Bonn, Germany, September 28 -- October 1 2021, Bonn, Germany.}
\mycvitem{}{\textbf{Moss, F. C.} \emph{Digitizing the Dualism Debate: a case study in the computational analysis of historical music theory sources}. CROSS 2021 Event. 16 September 2021, École Polytechnique Fédérale de Lausanne/Université de Lausanne, Lausanne, Switzerland.}
\mycvitem{}{\textbf{Moss, F. C.}, Herff, S. A., \& Rohrmeier, M. \emph{Modeling perceived tonal stability of individual and aggregated listener responses for scales and cadences}. 16th International Conference on Music Perception and Cognition \& 11th triennial conference of the European Society for the Cognitive Sciences of Music. July 28--31, Sheffield, UK [online].}
\mycvitem{}{\textbf{Moss, F. C.}, Herff, S. A., \& Rohrmeier, M. \emph{Individual perception of diatonic scales predicts perceived tonal fit in octatonic and hexatonic contexts}. 16th International Conference on Music Perception and Cognition \& 11th triennial conference of the European Society for the Cognitive Sciences of Music. July 28--31, Sheffield, UK [online].}
\mycvitem{}{Hentschel, J., \textbf{Moss, F. C.}, McLeod, A., \& Rohrmeier, M. \emph{Towards a Unified Model of Chords in Western Harmony}. Music Encoding Conference [online].}
\mycvitem{}{Anzuoni, E., Ayhan, S., Dutto, F., McLeod, A., \textbf{Moss, F. C.}, \& Rohrmeier, M. \emph{A Historical Analysis of Harmonic Progressions Using Chord Embeddings}. 18th Sound and Music Computing Conference [online].}
\mycvitem{}{\textbf{Moss, F. C.} \emph{Boosting Open Research in Empirical Musicology}. EPFL Data Champions Meeting (DCBreak\#3). March 18, 2021, Lausanne, Switzerland [online].}
\mycvitem{}{\textbf{Moss, F. C.} \emph{Discovering the line of fifths in a large historical corpus}. Future Directions of Music Cognition, The Ohio State University, March 6--7, 2021, Columbus, OH [online]. \url{https://doi.org/10.17605/OSF.IO/J5W6T}}

% 2020
\mycvitem{2020}{\inv\textbf{Moss, F. C.} \emph{The Importance of Modeling in Computational Musicology}. Round-table on ``Probability and Music'', 5th International Congress of Music and Mathematics (MusMat 2020) -- Perspectives and Applications of Mathematics in Post-Tonal Theories («Homage to Jamary Oliveira»),
    December 8--12, Rio de Janeiro, Brazil [online].}
\mycvitem{}{\textbf{Moss, F. C.} \emph{Analyzing musical pieces on the Tonnetz using the \emph{pitchplots} Python library}. 20. Jahreskongress der Gesellschaft für Musiktheorie
    (GMTH), Hochschule für Musik Detmold, October 1--4, 2020, Detmold, Germany [online].}
\mycvitem{}{\inv\textbf{Moss, F. C.} \emph{Computational Musicology and the Digital Humanities: Problems, Practices, and Prospects}. CRETA-Werkstatt \#9, Center for Reflected Text Analytics, University of Stuttgart, February 18, 2020, Stuttgart, Germany.}

% 2019
\mycvitem{2019}{\textbf{Moss, F. C.} \emph{Transitions of Tonality: Perspectives on the Historical Changes of Tonal Pitch Relations from Computational Musicology, Music Theory, and the Digital Humanities}. University of Cologne, November 29, 2019, Cologne, Germany.}
\mycvitem{}{\inv\textbf{Moss, F. C.} \emph{Tracing the History of Tonality with Note Distributions}. ``Corpus Research as a Means of Unlocking Musical Grammar'' International Research Workshop, July 1--4, 2019, Tel-Aviv, Israel.}
\mycvitem{}{\textbf{Moss, F. C.} \emph{Inferring Tonality from Note Distributions -- Why Models Matter (Poster)}. SEMPRE Graduate Conference 2019, Cambridge, UK.}
\mycvitem{}{\textbf{Moss, F. C.} \emph{Analyzing Tonality with Note Distributions}. First Swiss Digital Humanities Student Exchange DHX2019, Basel, Switzerland. }

% 2018
\mycvitem{2018}{\textbf{Moss, F. C.}, Souza, W. F. \& Rohrmeier, M. \emph{Harmony and Form in Brazilian Choro: A Corpus Study}. 15th International Conference on Music Perception and Cognition \& 10th triennial conference of the European Society for the Cognitive Sciences of Music, Graz, Austria. }
\mycvitem{}{Aitken, C., O'Donnell, T. \& Rohrmeier, M. [Poster presented by \textbf{Moss, F. C.}]. \emph{A Maximum Likelihood Model for the Harmonic Analysis of Symbolic Music}. 15th Sound and Music Computing Conference ``Sonic Crossings''. Limassol, Cyprus.}
\mycvitem{}{\inv\textbf{Moss, F. C.} \emph{Corpus Research in Digital Musicology}. Seminar ``Willkommen in der Matrix: Digitale Anwendungen f{\"u}r die Musikanalyse in Theorie und Praxis'', University of Basel, Basel, Switzerland.}
\mycvitem{}{Harasim, D., \textbf{Moss, F. C.} \& Ramirez, M. \emph{A Brief History of Tonality} (Poster). Applied Machine Learning Days, EPFL, Switzerland.}

% 2017
\mycvitem{2017}{\inv\textbf{Moss, F. C.} \emph{Formal Grammars and Ambiguity in Extended Tonality}. Workshop and Symposium on Schenkerian Analysis ``Wege der Kreativit{\"a}t -- Zwischen Erfindung und Rekonstruktion'', Universit{\"a}t der K{\"u}nste, Berlin, Germany.}
\mycvitem{}{\inv\textbf{Moss, F. C.} \emph{From Beethoven to Brazil: Digital Musicology at EPFL}. Digital Synergies: Ca' Foscari meets École Polytechnique Fédérale de Lausanne. Global Challenges Seminar - Team ``Creative arts, cultural heritage and digital humanities'', Venice, Italy.}
\mycvitem{}{\textbf{Moss, F. C.}, Souza, W. F. \& Rohrmeier, M. \emph{Brazilian Choro: A New Data Set of Chord Transcriptions and Analyses of Harmonic and Formal Features}. 17. Jahreskongress der Gesellschaft f{\"u}r Musiktheorie (GMTH) \& 27. Arbeitstagung der Gesellschaft f{\"u}r Popularmusikforschung (GfPM) ``Popul{\"a}re Musik und ihre Theorien: Begegnungen -- Perspektivwechsel -- Transfers'', Graz, Austria. }
\mycvitem{}{\textbf{Moss, F. C.}, Harasim, D., Neuwirth, M. \& Rohrmeier, M. \emph{Beethovens Streichquartette -- ein XML-basierter Korpus harmonischer Analysen in einem neuen Annotationssystem}. Jahrestagung der Gesellschaft f{\"u}r Musikforschung, Kassel, Germany.}
\mycvitem{}{\textbf{Moss, F. C.}, Rohrmeier, M. \emph{Integrating Transformational and Hierarchical Models of Extended Tonality}. 9th European Music Analysis Conference (EuroMAC), Strasbourg, France.}
\mycvitem{}{Rom, U., Je\ss ulat, A., \textbf{Moss, F. C.} \& Guter, I. \emph{Ambiguity, Illusion \& Timelessness in Late and Post-Tonal Harmony}. Panel discussion at the 9th European Music Analysis Conference (EuroMAC), Strasbourg, France.}
\mycvitem{}{\inv\textbf{Moss, F. C.} \emph{Musik und Sprache}. Talk for Student Association ``Denkzettel'', TUD, Dresden, Germany.}
\mycvitem{}{\textbf{Moss, F. C.}, Rohrmeier, M. \& Bravo, F. \emph{Emotional Associations Evoked by Structural Properties of Musical Scales and Abstract Visual Shapes}. KOSMOS Dialogue ``Music, Emotion, and Visual Imagery'', Berlin, Germany.}
\mycvitem{}{Harasim, D., \textbf{Moss, F. C.}, Neuwirth, M. \& Rohrmeier M. \emph{Beethoven's String Quartets: Introducing an XML-Based Corpus of Harmonic Labels Using a New Annotation System}. Music Encoding Conference, Tours, France.}

% 2016
\mycvitem{2016}{\inv\textbf{Moss, F. C.} \emph{Extended Tonality: Theoretical Challenges and their Relation to the Neuroscientific Study of Musical Syntax}. Max Planck Institute for Human Cognitive and Brain Sciences, Leipzig, Germany. }
\mycvitem{}{\textbf{Moss, F. C.}, Rohrmeier, M. \emph{Structural Ambiguities in Language and Music} (Poster). Helsinki Summer School for Cognitive Neuroscience 2016 (HSSCN 2016).}
\mycvitem{}{\textbf{Moss, F. C.}, Rohrmeier, M.  \emph{A grammatical approach to tension-resolution patterns in extended tonal harmony}. Meeting of the Computational Cognitive Science Group, Massachusetts Institute of Technology, Department of Brain and Cognitive Sciences, Cambridge, USA.}
\mycvitem{}{\inv\textbf{Moss, F. C.}, Rohrmeier, M. \emph{Towards a syntactic account for harmonic sequences in extended tonality}. Syntax Square Meeting, Massachusetts Institute of Technology, Department of Linguistics and Philosophy, Cambridge, USA.}
\mycvitem{}{\textbf{Moss, F. C.} \emph{Syntax of Extended Tonality: Towards a Grammar of Generalized Harmonic Functions}. Music Theory Colloquium, Boston University, College of Fine Arts, School of Music, Boston, USA.}
\mycvitem{}{\textbf{Moss, F. C.} \emph{Generalizing Harmonic Functions: A Grammatical Approach to Extended Tonality}. Yale University, Department of Music, New Haven, USA.}
\mycvitem{}{\inv\textbf{Moss, F. C.} \& Harasim, D. \emph{Extended Tonality and Music Cognition}. Symposium ``Towards a World Music Theory'', University of Hamburg, Institute for Systematic Musicology, Hamburg, Germany.}
\mycvitem{}{\textbf{Moss, F. C.} \emph{Music Cognition and Extended Tonality: Theoretical Challenges and Empirical Implications}. Research Colloquium, University of Cologne, Cologne, Germany.}

% 2015
\mycvitem{2015}{\textbf{Moss, F. C.} \emph{On generative modelling of musical form}. Seminar ``Mathematics and Music'', TUD, Dresden, Germany.}
\mycvitem{}{\textbf{Moss, F. C.} \emph{`The terror of sanctity.' Tonal cues for resolving dramatic ambiguities in Wagner's Parsifal}. Seminar ``Understanding Musical Structures'', TUD, Dresden Germany.}

% 2014

\mycvitem{2014}{\textbf{Moss, F. C.} \emph{Tonality and functional equivalence: A multi-level model for the cognition of triadic progressions in 19th century music}. International conference of Students of Systematic Musicology, Goldsmiths University, London, UK.}
\mycvitem{}{\textbf{Moss, F. C.} \emph{Language, music and the brain: a resource-sharing framework (Patel, 2012)}. Seminar ``Cognitive Neuroscience of Music'', Institut for Musicology, University of Cologne, Cologne, Germany.}