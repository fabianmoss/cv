%----------------------------------------------------------------------------------------
%	PUBLICATIONS SECTION
%----------------------------------------------------------------------------------------

\section{Publications \textcolor{black}{\small(\OA = Open Access)}}

\subsection{Theses}

\mycvitem{PhD}{\OA\textbf{Moss, F. C.} (2019). \emph{Transitions of Tonality: A Model-Based Corpus Study}. Doctoral dissertation. École Polytechnique Fédérale de Lausanne, Lausanne, Switzerland.
    Supervisors: Martin Rohrmeier \& Markus Neuwirth. \url{https://doi.org/10.5075/epfl-thesis-9808}}

\mycvitem{MA}{\textbf{Moss, F. C.} (2012). \emph{``Theorie der Tonfelder'' nach Simon und ``Neo-Riemannian Theory'':
    Systematik, historische Bez{\"u}ge und analytische Praxis im Vergleich}. Supervisor: Hans Neuhoff. \url{https://doi.org/10.5281/zenodo.4748512}}

\subsection{Journal Articles and Conference Papers}

%% In Press

%% Submitted
% \mycvitem{}{\textbf{Moss, F. C.}, Herff, S., \& Rohrmeier, M. (submitted). Evidence for cognitive tonal hierarchies in cadential but not scalar contexts.} %  \emph{Music \& Science}.

%% In Review

%% Accepted
\mycvitem{accepted}{\OA Rohrmeier, M., \& \textbf{Moss, F. C.} (accepted). A Formal Model of Extended Tonal Harmony
% from 19th Century to Jazz, Film, and Pop Music. 
In: \emph{Proceedings of the 22nd International Society for Music Information Retrieval Conference.} [Online].}
\mycvitem{}{\OA Hentschel, J., \textbf{Moss, F. C.}, \& Rohrmeier, M. (accepted). A Semi-Automated Workflow Paradigm for the Distributed
Creation and Curation of Expert Annotations. In: \emph{Proceedings of the 22nd International Society for Music Information Retrieval Conference.} [Online].}
\mycvitem{}{\OA \textbf{Moss, F. C.}, Köster, M., Femminis, M., Metrailler, C., \& Bavaud, F. (accepted). Digitizing a 19th-century music theory debate for computational analysis. In: \emph{CHR 2021: Computational Humanities Research Conference}, 
    Amsterdam, The Netherlands.}
\mycvitem{}{\OA \textbf{Moss, F. C.} \& Rohrmeier, M. (accepted). Discovering tonal structures with Latent Dirichlet Allocation. \emph{Music \& Science}. \url{https://doi.org/10.1177/20592043211048827}}
\mycvitem{}{\OA \textbf{Moss, F. C.}, Neuwirth, M., \& Rohrmeier, M. (accepted). The line of fifths and the co-evolution of tonal pitch-classes. \emph{Journal of Mathematics and Music}.}
\mycvitem{}{\OA \textbf{Moss, F. C.} \& Neuwirth, M. (accepted). FAIR, Open, Linked: Introducing the Special Issue on Open Science in Musicology. \emph{Empirical Musicology Review} 16(1).}
\mycvitem{}{\OA Viaccoz, C., Harasim, D., \textbf{Moss, F. C.}, \& Rohrmeier, M. (accepted). Wavescapes: A Visual Hierarchical Analysis of Tonality Using the Discrete Fourier Transformation. \emph{Musicae Scientiae}. \url{https://doi.org/10.1177/10298649211034906}}

% In Preparation
% \mycvitem{in prep.}{\textbf{Moss, F. C.}, Hentschel, J., Neuwirth, M., \& Rohrmeier, M. (in prep.). The harmonic vocabulary of 19th-century piano composers.}
% \mycvitem{}{\textbf{Moss, F. C.}, Herff, S., \& Rohrmeier, M. (in prep.). Individual perception of diatonic scales predicts perceived tonal fit in extended tonality.} %  \emph{Psychonomic Bulletin \& Review}.
% \mycvitem{}{\textbf{Moss, F. C.}, Noll, T., \& Rohrmeier, M. (in prep.). Surfing the Chromatic Waves: Detecting Tone Fields Using Discrete Fourier Analysis.}
% \mycvitem{}{\textbf{Moss, F. C.}, Lieck, R., \& Rohrmeier, M. Tracing Historical Changes in Tonality with the Tonal Diffusion Model.}
% \mycvitem{}{\textbf{Moss, F. C.} Polytonality and the Emergence of Tone Fields in Tailleferre's \emph{Pastorale} (1919).}
% \mycvitem{}{Métrailler, C., Bavaud, F., \& \textbf{Moss, F. C. } CROSS-Methods}
% \mycvitem{}{Köster, M., Bavaud, F., \& \textbf{Moss, F. C. } CROSS-Close reading}

%% 2021

\mycvitem{2021}{\OA Hentschel, J., \textbf{Moss, F. C.}, McLeod, A., Neuwirth, M., \& Rohrmeier, M. (2021). Towards a Unified Model of Chords in Western Harmony. In: \emph{Music Encoding Conference Proceedings 2021}. Alicante, Spain [Online].}
\mycvitem{}{\OA Anzuoni, E., Ayhan, S., Dutto, F., Mcleod, A., \textbf{Moss, F. C.}, \& Rohrmeier, M. (2021). A Historical Analysis of Harmonic Progressions Using Chord Embeddings. \emph{Proceedings of the 18th Sound and Music Computing Conference}, 284–291. \url{https://doi.org/10.5281/zenodo.5038910}}
\mycvitem{}{\OA Harasim, D., \textbf{Moss, F. C.}, Ramirez, M., \& Rohrmeier, M. (2021). Exploring the foundations of tonality: Statistical cognitive modeling of modes in the history of Western classical music. \emph{Humanities and Social Sciences Communications}, 8(5), 1--11. \url{https://doi.org/10.1057/s41599-020-00678-6}}

%% 2020
\mycvitem{2020}{\OA Lieck, R., \textbf{Moss, F. C.}, \& Rohrmeier, M. (2020). The Tonal Diffusion Model. \emph{Transactions of the Inter\-national Society of Music Information Retrieval}, 3(1), 153--164. \url{https://doi.org/10.5334/tismir.46}}
\mycvitem{}{\OA \textbf{Moss, F. C.}, de Souza, W. F., \& Rohrmeier, M. (2020). Harmony and Form in Brazilian Choro: A Corpus-Driven Approach to Musical Style Analysis. \emph{Journal of New Music Research},  49(5), 416--437. \url{https://doi.org/10.1080/09298215.2020.1797109}}

%% 2019
\mycvitem{2019}{\OA \textbf{Moss, F. C.}, Neuwirth, M., Harasim, D., \& Rohrmeier, M. (2019). Statistical characteristics of tonal harmony: A corpus study of Beethoven's string quartets. \emph{PLOS ONE}, 14(6), e0217242. \url{https://doi.org/10.1371/journal.pone.0217242}}
\mycvitem{}{\OA Landnes, K., Mehrabyan, L., Wiklund, V., Lieck, R., \textbf{Moss, F. C.}, \& Rohrmeier, M. (2019). A
    Model Comparison for Chord Prediction on the Annotated Beethoven Corpus. In I.
    Barbancho, L. J. Tard{\'o}n, A. Peinado, \& A. M. Barbancho (Eds.), \emph{Proceedings of the 16th Sound and Music Computing Conference (SMC 2019)} (pp. 250--254). M{\'a}laga, Spain.}
\mycvitem{}{\OA Popescu, T., Neuser, M. P., Neuwirth, M., Bravo, F., Mende, W., Boneh, O., \textbf{Moss, F. C.}, \& Rohrmeier, M. (2019). The pleasantness of sensory dissonance is mediated by musical style and expertise. \emph{Scientific Reports}, 9(1), 1070. \url{https://doi.org/10.1038/s41598-018-35873-8}}

%% 2018
\mycvitem{2018}{\OA Neuwirth, M., Harasim, D., \textbf{Moss, F. C.}, \& Rohrmeier, M. (2018). The Annotated Beethoven Corpus~(ABC): A Dataset of Harmonic Analyses of All Beethoven String Quartets. \emph{Frontiers in Digital Humanities}, 5(July), 1--5. \url{https://doi.org/10.3389/fdigh.2018.00016}}

%% Earlier
\mycvitem{earlier}{\textbf{Moss, F. C.} (2017). [Review of David Huron. Voice Leading: The Science behind a Musical Art]. Music Theory \& Analysis, 4(1), 119--130. \url{https://doi.org/10.11116/MTA.4.1.7}}
\mycvitem{}{\OA \textbf{Moss, F. C.} (2014). Tonality and functional equivalence: A multi-level model for the cognition of triadic progressions in 19th century music. In \emph{International Conference of Students of Systematic Musicology -- Proceedings} (pp. 1--8). London, UK.}
\mycvitem{}{\OA \textbf{Moss, F. C.} \emph{Albert Simons \emph{Theorie der Tonfelder} und John Cloughs \emph{Flip-Flop Circles} im Vergleich}. Zenodo. \url{http://doi.org/10.5281/zenodo.3944462}}

\subsection{As editor}
\mycvitem{forthcoming}{Special Issue on ``Open Science in Musicology'' in \emph{Empirical Musicology Review}, with Markus Neuwirth.}

\subsection{Datasets and Code}

\mycvitem{2020}{\OA \textbf{Moss, F. C.}, Neuwirth, M., \& Rohrmeier, M. (2020). Tonal Pitch-Class Counts Corpus (TP3C) [Data set]. \emph{Zenodo}. \url{https://doi.org/10.5281/zenodo.3600080}}
\mycvitem{2019}{\OA \textbf{Moss, F. C.}, Loayza, T., \& Rohrmeier, M. (2019). pitchplots. \emph{Zenodo}. \url{https://doi.org/10.5281/zenodo.3265392}}
\mycvitem{2018}{\OA \textbf{Moss, F. C.}, de Souza, W. F., \& Rohrmeier, M. (2018). Choro Songbook Corpus [Data set]. \emph{Zenodo}. \url{https://doi.org/10.5281/zenodo.1442764}}

\subsection{Blogposts}

\mycvitem{2020}{``A computational model for note distributions in musical pieces''\newline \url{https://www.epfl.ch/labs/dcml/computational-model-note-dists/}}
\mycvitem{}{``Tracing historical changes in the exploration of tonal space''\newline \url{https://www.epfl.ch/labs/dcml/tracing-historical-changes/}}