\section{Service}

\subsection{Memberships}

\textbf{06/2023--present} University of Wuerzburg \emph{Graduate School Humanities}, classes ``Digital Humanities'' \& ``Philosophy, Languages, Arts''
\textbf{12/2022--present} Deutsche Gesellschaft Juniorprofessur (DGJ); Arbeitskreis ``Philologie und Digitalität'' (JMU)
\textbf{10/2022--present} European COST Action \emph{EarlyMuse} (2022--2026), Working Groups 2 (Sources) \& 3 (Publications) \url{https://www.cost.eu/actions/CA21161/}
\textbf{11/2021--present} Deutscher Hochschulverband (DHV);
\textbf{11/2021--present} European Society for the Cognitive Sciences of Music (ESCOM);
\textbf{09/2021--present} International Society for Music Information Retrieval (ISMIR);
\textbf{05/2020--12/2021} EPFL Data Champions Community;
\textbf{10/2019--present} Gesellschaft f{\"u}r Musikforschung (GfM);
\textbf{10/2018--present} Gesellschaft f{\"u}r Musiktheorie (GMTH);
\textbf{02/2018--12/2022} UNIL-EPFL Centre for Digital Humanities (dhCenter);
\textbf{03/2015--08/2017} Dresden Technical University Graduate Academy

\subsection{Responsibilities}

\textbf{08/2023--present} Scientific advisory board of \textit{Corpus Monodicum} project
\textbf{2023} Program Committee Member for International Conference on Multimedia
Retrieval
\textbf{2022} Scientific Committee for Workshop on Computational Methods in the Humanities 2022 (COMHUM 2022);
\textbf{2021} Programm Committee for 2nd Conference on Computational Humanities Research (CHR2021);
\textbf{since 2020} Co-Chair of the Music Analysis Interest Group of the \emph{Music Encoding Initiative} (MEI);
\textbf{2018--2019} Co-founder and vice-president of the Digital Humanities Student Association \emph{dhelta} at EPFL;
\textbf{2012/10--2013/09} Financial officer for General Students' Commitee, HfMT

\subsection{Reviewer activity}

\mycvitem{Journals}{
    \emph{Digital Scholarship in the Humanities};
    \emph{Empirical Musicology Review};
    \emph{Music and Science};
    \emph{Music Theory and Analysis};
    \emph{Royal Society Open Science};
    \emph{Transactions of the International Society of Music Information Retrieval};
    \emph{Zeitschrift der Gesellschaft für Musiktheorie % [Journal of the German-speaking Society of Music Theory]
    }
}
\mycvitem{Conferences}{
    \emph{Computational Humanities Research~(CHR)};
    \emph{Conference of the European Society for the Cognitive Sciences of Music~(ESCOM)};
    \emph{International Conference on Multimedia Retrieval (ICMR)};
    \emph{International Conference on Music Perception and Cognition~(ICMPC)};
    \emph{International Conference of Students of Systematic Musicology (SysMus)};
    \emph{Jahrestagung der Gesellschaft für Informatik (GI), Workshop zu Informatik und Digital Humanities (InfDH)}
}

\subsection{Organization}

% \mycvitem{2024}{Workshop on Bayesian modeling for music research with Christoph Finkensiep (Universiteit van Amsterdam) and Andrew McLeod (Fraunhofer Institute for Digital Media Technology), January--February, 2024. Zentrum für Philologie und Digitalität (ZPD), JMU, Würzburg, Germany.}
\mycvitem{2023}{Lecture series ``CODAMUS: Computational and Digital Approaches to Music Scholarship'', 18 October, 2023--07 February, 2024. Zentrum für Philologie und Digitalität (ZPD), JMU, Würzburg, Germany. \url{https://codamus.pubpub.org/}}
\mycvitem{}{Kontrapunkt-Werkstatt ``Latest Tools for Analyzing Early Music'', with Hansjörg Ewert, Florian Vogt, \& Ugo Bindini, 20--21 October, Würzburg, Germany. \url{https://www.musikwissenschaft.uni-wuerzburg.de/diversa/tagungen/basel23/}}
\mycvitem{}{Open project space for 16 contributions ``Methoden und Ziele digitaler Musikwissenschaft: Ein Marktplatz aktueller Forschung'', with Stefanie Acquavella-Rauch, Martin Albrecht-Hohmeier, Irmlind Capelle, Jürgen Diet, \& Jens Dufner. Jahrestagung der Gesellschaft für Musikforschung, 4--7 October, 2023, Saarbrücken, Germany.\\\url{https://www.uni-saarland.de/methoden-und-ziele-digitaler-musikwissenschaft-ein-marktplatz-aktueller-forschung.html}}
\mycvitem{2022}{CREATE Salon on ``Computational Creativity'', 23 November 2022, \emph{Creative Amsterdam: An E-Humanities Perspective}, Media Studies Department, University of Amsterdam, The Netherlands. \url{https://www.create.humanities.uva.nl/events/computational-creativity/}}
\mycvitem{}{Workshop ``Representing Harmony: Goals and Challenges'', with Johannes Hentschel, Markus Neuwirth \& Martin Rohrmeier. 13--16 September 2022, Digital and Cognitive Musicology Lab, École Polytechnique Fédérale de Lausanne, Switzerland. \url{https://www.epfl.ch/labs/dcml/workshops/representing-harmony/}}
\mycvitem{2021}{Workshop ``Musik -- Schrift -- Digitalität'' [Music -- Writing -- Digitality], with Dennis Ried and Daniel Fütterer. 13--14 December 2021, Hochschule für Musik, Karlsruhe, Germany.}
\mycvitem{2019}{Workshop ``Schenkerian and Tonfeld Theory for Music Analysis''. 12--15 December 2019, Digital and Cognitive Musicology Lab, École Polytechnique Fédérale de Lausanne, Switzerland.\\ \url{https://memento.epfl.ch/event/masterclass-schenkerian-and-tonfeld-theory-for-mus/}}
\mycvitem{}{First Swiss Digital Humanities Exchange, with Jessica Pidoux, Gerhad Lauer, and Stefan Münnich. 8--9 February 2019, DH Lab, University of Basel, Switzerland. \url{https://sites.google.com/view/dhexchange/}}
\mycvitem{2015}{Co-organization of lecture series ``Systematic Musicology: Perception and Cognition of Music'', lead: Martin Rohrmeier. Dresden Music Cognition Lab, Technichal University Dresden, Germany.}
\mycvitem{2013}{Co-organization of the international conference ``Musical Metre in Comparative Perspective'', lead: Hans Neuhoff and Rainer Polak. 4--6 April 2013, Hochschule für Musik und Tanz Köln, Germany.}
