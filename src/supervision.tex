\section{Supervision and mentoring}

\subsection{PhD thesis supervision}
%\mycvitem{2025/05--today}{Sebina Weich: ``Musical form in the Renaissance. Mode as a form-constituting element in Pontio's treatises'' (Music Theory), JMU}
\mycvitem{Current}{Adrian Nachtwey: 
    ``Eine Studie zur textkritischen Analyse von Musikeditionsvarianten 
    im 19. Jahrhundert unter Anwendung von digitalen Methoden'' (Musicology), JMU} % 10/2023--today
\mycvitem{}{Tim Eipert: ``A Quantitative Perspective on Transmission, Structure, and Modality of Medieval Chant'', Graduate School Humanities (Digital Humanities), JMU}
%\mycvitem{}{Lucas Hofmann: ``Computational modeling of complex temporal and tonal structures in early twentieth-century music'', Graduate School Humanities (Digital Humanities), JMU}
\mycvitem{}{Silas Bischoff: ``Aufstieg und Fall der Deutschen Lautentabulatur -- Eine Untersuchung zu ihrem Ursprung und zu ihrer Entwicklungsgeschichte'' (Musicology), JMU (1st supervisor: Ulrich Konrad)} % 2024--today

\mycvitem{Past}{Shuxin Meng: ``Mapping the Structure of Chinese Folk Songs: From Corpus Analysis to Cross-Cultural Listening'', Digital Humanities, EPFL (1st supervisor: Martin Rohrmeier)} % 07/2022--today
\mycvitem{}{Willian Fernandes de Souza (peer-mentoring):
    ``Estilo e Sintaxe: quatro ensaios analíticos em práticas do choro'' (Music Theory/Composition), Universidade Federal do Rio de Janeiro (UFRJ)} % Spring 2017

\subsection{Master thesis supervision}
\mycvitem{Spring 2024}{Zihan Guo: ``Die Entwicklung der Interpretationen von Brahms' A-Dur Violinsonate Op. 100 in den 1930er bis 1970er Jahren: Eine Analyse historischer Aufnahmen unter Anwendunge moderner Analysetools'' (Musicology), JMU
}
\mycvitem{Fall 2023}{%
    Francesco Paolo Leonardo La Barbera: ``Proportionen, Transformationen oder Tonfelder? Die vergleichende Anwendung dreier musiktheoretischer Ansätze'' (Musicology), Universität Leipzig (1st supervisor: Stefan Keym)}
\mycvitem{}{Felicitas Stickler: ``Das Passionsoratorium „Der sterbende Heiland“ von Ignaz Franz Xaver Kürzinger. Edition -- Kritischer Bericht -- Analytische Aspekte'' (Musicology), JMU (1st supervisor: Ulrich Konrad)}
\mycvitem{}{Julia Groblewski-Meiser: ``Narration und Interpretation:  Allegorische Darstellungen einer musikalischen Harmonie im Kuppelfresko von Santa Maria del Fiore von Giorgio Vasari'' (Musicology), JMU}
\mycvitem{}{Oscar Aquite Pena: ``Between millo and picó: music as discursive masking in
 \emph{La Puntica No Ma'}, costume troupe of the Barranquilla Carnival (Colombia)'' (Ethnomusicology), JMU (1st supervisor: Nepomuk Riva)}
\cvitem{Spring 2020}{Cédric Viaccoz (Digital Humanities, 3rd supervisor): ``Visual Hierarchical Analysis of Tonality using the
Discrete Fourier Transform'', EPFL}

\subsection{Bachelor thesis supervsion}
\mycvitem{Fall 2024}{Miriam Fodil: ``Die Rolle parasozialer Beziehungen in der Entwicklung von Fan-Economies: Wie K-Pop-Unternehmen von emotionalen Fanbindungen profitieren'', Musicology, JMU.'}
\mycvitem{}{Felicitas Stickler: ``Ein minimales Modell für die gemischte Kodierung von Text (TEI) und Musiknotation (MEI)'', Digital Humanities, JMU}
\mycvitem{Spring 2023}{Corinna Bongartz: 
``Musik und Künstliche Intelligenz: Eine Untersuchung der Zuordnung 
festgelegter Prompts zu durch Sprachmodellen erzeugt Musiksnippets'', 
Musicology, JMU}
\mycvitem{Spring 2022}{Iris Folpmers (2nd supervisor): ``Data Sonification: Turning Climate Data into
Music'' Artificial Intelligence, UvA, \url{https://scripties.uba.uva.nl/search?id=record_29490}}

\subsection{Other mentoring}
\cvitem{Fall 2024}{Digital-Humanities Projekt ``Digitale Präsentation von XML-kodierten musiktheoretischen Texten mit CETEIcean'' (Felicitas Stickler)}
\cvitem{Fall 2020}{3 Machine Learning graduate student projects on vector embeddings of harmony (EPFL)}
\cvitem{Fall 2019}{Machine Learning graduate student project on vector embeddings of harmony (EPFL)}
\cvitem{Fall 2018}{3 Machine Learning graduate student projects on chord prediction with neural networks (EPFL)}
\cvitem{Spring 2018}{4 Digital Musicology graduate student projects (EPFL)}
\cvitem{Fall 2015}{interdisciplinary project of technical design undergraduate, Technische Universität Dresden (TUD)}
