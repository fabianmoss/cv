\section{Organization}

%\mycvitem{2022}{CREATE Salon on ``Computational Creativity'' (invited speakers: Lev Manovich, Martin Rohrmeier, \& XXX), 18 October 2022, \emph{Creative Amsterdam: An E-Humanities Perspective}, Media Studies Department, University of Amsterdam, The Netherlands.}
\mycvitem{2022}{Workshop ``Representing Harmony: Goals and Challenges'', with Johannes Hentschel, Markus Neuwirth \& Martin Rohrmeier. 13--16 September 2022, Digital and Cognitive Musicology Lab, École Polytechnique Fédérale de Lausanne, Switzerland.}
\mycvitem{2021}{Workshop ``Musik -- Schrift -- Digitalität'' [Music -- Writing -- Digitality], with Dennis Ried and Daniel Fütterer. 13--14 December 2021, Hochschule für Musik, Karlsruhe, Germany.}
\mycvitem{2019}{Workshop ``Schenkerian and Tonfeld Theory for Music Analysis''. 12--15 December 2019, Digital and Cognitive Musicology Lab, École Polytechnique Fédérale de Lausanne, Switzerland.}
\mycvitem{}{First Swiss Digital Humanities Exchange, with Jessica Pidoux, Gerhad Lauer and Stefan Münnich. 8--9 February 2019, DH Lab, University of Basel, Switzerland.}
\mycvitem{2015}{Co-organization of lecture series ``Systematic Musicology: Perception and Cognition of Music'', lead: Martin Rohrmeier. Dresden Music Cognition Lab, Technichal University Dresden, Germany.}
\mycvitem{2013}{Co-organization of the international conference ``Musical Metre in Comparative Perspective'', lead: Hans Neuhoff and Rainer Polak. 4--6 April 2013, Hochschule für Musik und Tanz Köln, Germany.}

\section{Service}

\subsection{Reviewer activity}
\mycvitem{Journals}{
    \emph{Digital Scholarship in the Humanities},
    \emph{Empirical Musicology Review},
    \emph{Music and Science},
    \emph{Music Theory and Analysis},
    \emph{Transactions of the International Society of Music Information Retrieval},
    \emph{Zeitschrift der Gesellschaft für Musiktheorie % [Journal of the German-speaking Society of Music Theory]
    }
}

\mycvitem{Conferences}{
    \emph{Computational Humanities Research~(CHR)},
    \emph{Conference of the European Society for the Cognitive Sciences of Music~(ESCOM)},
    \emph{International Conference on Music Perception and Cognition~(ICMPC)},
    \emph{International Conference of Students of Systematic Musicology (SysMus)}
}

\subsection{Responsibilities and memberships}
\mycvitem{2022}{Member of European COST Action \emph{EarlyMuse} (2022--2026), Working Groups 2 (Sources) \& 3 (Publications) \url{https://www.cost.eu/actions/CA21161/}}
\mycvitem{}{Scientific Committee Member for Workshop on Computational Methods in the Humanities 2022 (COMHUM 2022), 9--10 June, 2022, Lausanne, Switzerland.}
\mycvitem{2021}{Programm Committee Member for 2nd Conference on Computational Humanities Research (CHR2021), 17--19 November, 2021, Amsterdam, The Netherlands [online].}
\mycvitem{since 2021}{European Society for the Cognitive Sciences of Music (ESCOM), International Society for Music Information Retrieval (ISMIR)}
\mycvitem{since 2020}{Co-Chair of the Music Analysis Interest Group of the \emph{Music Encoding Initiative} (MEI).}
\mycvitem{since 2019}{UNIL-EPFL Centre for Digital Humanities (dhCenter); EPFL Data Champions Community;
    Gesellschaft f{\"u}r Musikforschung (GfM).}
\mycvitem{2018--2019}{Co-founder and vice-president of the Digital Humanities Student Association \emph{dhelta} at EPFL.}
\mycvitem{since 2017}{Gesellschaft f{\"u}r Musiktheorie (GMTH).}
\mycvitem{2012/10--2013/09}{Financial officer for General Students' Commitee [Finanzreferent AStA], HfMT.}